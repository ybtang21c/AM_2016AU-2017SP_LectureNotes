\begin{center}
	{\bf \Large 高等数学(上)复习要点}
	
	\vspace{1em}
	
	{\it(免责声明:以下内容仅供复习参考,与考试内容若有雷同,纯属巧合)}
	
	{\it updateded: 2015-12-29}
\end{center}

\setlength{\parindent}{-10pt}

{\bf 一、集合与函数}
\begin{enumerate}
%   \item 常用不等式
%   \begin{itemize}
%     \item 绝对值不等式:$||a|-|b||\leq |a-b|\leq |a|+|b|$
%     \item 平均值不等式:$\sqrt[n]{\prod\limits_{k=1}^n}a_k
%     \leq\df 1n{\sum\limits_{k=1}^n}a_k
%     \quad (a_k\geq 0)$
% %     \item Bernoouli不等式:$(1+h)^n\geq 1+nh,\;(h\geq -2,\,n\in\mathbb{N})$
% 	\item Cauchy不等式:$\left(\sum\limits_{i=1}^na_ib_i\right)\leq
% 	\sum\limits_{i=1}^na_i^2\sum\limits_{i=1}^nb_i^2$
% 	\item Schwarz积分不等式:$\left[\dint_a^bf(x)g(x)\d x\right]^2
% 	\leq\dint_a^bf\,^2(x)\d x\dint_a^bg^2(x)\d x$ 
%   \end{itemize}
  \item 初等函数
  \begin{itemize}
%     \item 基本初等函数:幂函数、指数函数、对数函数、三角函数、反三角函数
    \item 注意反三角函数的定义域与值域:
    \begin{itemize}
      \item $\arcsin x:[-1,1]\to[-\pi/2,\pi/2]$
      \item $\arccos x:[-1,1]\to[0,\pi]$
      \item $\arctan x:\mathbb{R}\to[-\pi/2,\pi/2]$
      \item 例:分别讨论$\sin(\arcsin x)$与$\arcsin(\sin x)$的定义域与值域
    \end{itemize}
    \item 熟练掌握三角函数积化和差、和差化积公式,半角/倍角公式,万能公式
  \end{itemize}
%     \begin{itemize}
%       \item $\sin A + \sin B=2\sin\df{A+B}{2}\cos\df{A-B}2$
%       \item $\sin A\sin B=\df{\cos (A-B)-\cos(A+B)}{2}$
%       \item 记$t=\tan\df x2$,则$\sin x=\df{2t}{1+t^2},\cos
%       x=\df{1-t^2}{1+t^2},\tan x=\df{2t}{1-t^2}$
%     \end{itemize}
% 	\item 函数的基本性质
% 	\begin{itemize}
%       \item 单调性、有界性、奇偶性、对称性、周期性的定义
%       \item 反面说法:1)“与”和“或”交换;2)“存在”和“任意”互换;3)“大(小)于”和“小(大)于等于”互换,例如:
%       \begin{itemize}
%         \item $f(x)$在区间$[a,b]$上有界:存在$M>0$,对任意$x\in[a,b]$,有
%         $|f(x)|<M$
%         \item (反面说法)$f(x)$在区间$[a,b]$上无界:对任意$M>0$,总存在$x_0\in[a,b]$,使得
%         $|f(x_0)|\geq M$
%       \end{itemize}
% 	\end{itemize}
    \item 常用函数
    \begin{itemize}
      \item 符号函数:$\mathrm{sgn}\,x$,重要性质:$|x|=x\cdot\mathrm{sgn}\,x$
%       \item 取整函数(阶梯函数):$[\,x\,]$
      \item Dirichlet函数:$D(x)$,掌握如下证明:
      \begin{itemize}
	    \item $xD(x)$只在$x=0$连续(习题3.4-14)
	    \item $D(x)$不可积(P309-例2)
	  \end{itemize}
%       \item Riemann函数(了解即可):$R(x) =\left\{
% 		\begin{array}{ll}
% 		1,\;&x=0\\
% 		\displaystyle\frac 1q,\;&x=\displaystyle\frac pq,\,p,q\mbox{为互素的整数}\\
% 		0,\;&x\notin\mathbb{Q}
% 		\end{array}
% 	  \right. \;x\in[0,1]$
% 	  \begin{itemize}
% 	    \item $R(x)$对任意$x_0\in[0,1]$,$\lim\limits_{x\to x_0}R(x)=0$
% 	    \item $R(x)$在无理数点连续,有理数点不连续
% 	  \end{itemize}
    \end{itemize}
    \item 曲线的参数方程
    \begin{itemize}
      \item 熟练掌握极坐标与平面直角坐标的相互转换
      \item 注意:使用参数方程时,必须明确指出参数的变化范围!
    \end{itemize}
\end{enumerate}

\bigskip

{\bf 二、数列极限}
\begin{enumerate}
  \item 掌握用极限的$\e-N$定义证明极限的方法
  \item 极限的性质\ps{注意类比复习函数极限的有关性质}
  \begin{itemize}
    \item 掌握极限保号性的不同叙述(各种推论)和证明方法(P53-定理2.1.3,习题2.1-9)
  \end{itemize}
  \item 收敛的判定
  \begin{itemize}
    \item 单调有界原理:多用于证明递推数列的收敛性
    \item 子数列的收敛性
    \begin{itemize}
      \item 掌握子数列收敛的$\e-N$定义!
      \item 多用于证明原数列的极限不存在
    \end{itemize}
  \end{itemize}
  \item 极限的计算:参见“求极限的常用方法及典型例题”(重点)
  \item 递推数列的极限(难点)
  \begin{itemize}
    \item 能通过递推式求出极限的,可考虑先使用单调有界原理证明极限存在,
    再在递推式两端同时取极限,解方程求出极限值(P65-例9)
    \item 不能通过递推式求出极限的,则必须通过递推式求出数列的通项表达式(习题2.2-8)
  \end{itemize}
\end{enumerate}

\bigskip

{\bf 三、无穷级数(难点)}
\begin{enumerate}
  \item 级数收敛的定义
  \begin{itemize}
    \item 级数$\sumn a_n$收敛等价于其部分和数列$\{S_n\}$收敛:$\sumn a_n=\limn S_n$
    \item 与反常积分类比:反常积分收敛等价于对应变限积分的极限存在
    \begin{itemize}
      \item 无穷积分:
      $\dint_a^{+\infty}f(x)\d x=\lim\limits_{t\to+\infty}\dint_a^tf(x)\d x$
%       \item 瑕积分(设$a$为$f(x)$的瑕点):
%       $\dint_a^bf(x)\d x=\lim\limits_{t\to a^+}\dint_t^bf(x)\d x$
    \end{itemize}
    \item 收敛的必要条件:$\limn a_n=0$,主要用于证明级数发散
    \begin{itemize}
      \item 类比:$\dint_a^{+\infty}f(x)\d x$,则必有$\limx{+\infty}f(x)=0$
    \end{itemize}
  \end{itemize} 
  \item 正项级数收敛性的判定:\ps{({\bf 注意:}(1)以下判别法只对正项级数成立;
  (2)如无特别说明,判别法均为充分条件)}
  \begin{itemize}
    \item 通常首先尝试“比/根植判别法”
    \begin{itemize}
      \item 注意不等式形式与极限形式在叙述上的不同:例如:比值判别法的不等式形式:
      设$a_n\geq 0(n\in\mathbb{N})$,且当$n$充分 大时,总有$\df{a_{n+1}}{a_n}\leq
      q<1$,则$\sumn a_n$收敛。注意,其中要求两项的比值必须 小于等于某个比$1$小的正数$q$,
      而不是仅仅要求$\df{a_{n+1}}{a_n}<1$,注意区别。
    \end{itemize}
    \item 接下来可以考虑“比较法(不等式形式与极限形式)”:最有用也最难用
    \begin{itemize}
      \item 常见的比较对象:$p$-级数、调和级数、几何级数
    \end{itemize}
    \item 有些特殊情形可以用到收敛的充要条件:部分和有界,通常需要用到有理分式的分解(见附录)
    \item 了解Raabe判别法,不推荐使用!
  \end{itemize}
  \item 交错级数收敛的Leibnitz判别法:是充分而非必要条件!!
  \item 变号级数收敛性
  \begin{itemize}
    \item 绝对收敛必收敛
    \item 掌握条件收敛与绝对收敛的关系
  \end{itemize}
  \item 无穷级数的积分判别法:
  \begin{itemize}
    \item 若$x\to +\infty$时,$f(x)$单调趋于零,则$\sumn
    f(n)$与$\dint_a^{+\infty}f(x)\d x$同敛散
  \end{itemize}
\end{enumerate}

\bigskip

{\bf 四、函数的极限与连续性}
\begin{enumerate}
  \item 函数极限
  \begin{itemize}
    \item 掌握用极限的$\e-\delta$定义证明极限的方法
    \item 注意:$\limx{x_0}f(x)$与$f(x_0)$无关
    \item 掌握极限保号性的不同叙述(推论)、证明方法(P110-定理3.1.4)及其应用(习题6.1-6)
    \item 掌握极限$\limx{0}\df{\sin
      x}{x}=1$和$\limx{\infty}\left(1+\df 1x\right)^x=e$(P120-121)的证明
    \item Heine定理:常用于证明极限不存在(要求能正确理解并能完整叙述该定理)
  \end{itemize}
  \item 函数的连续性
  \begin{itemize}
    \item 熟练掌握间断点的分类(常见于填空和判断题,讨论铅直渐近线时也有应用)
    \item 有界闭区间上连续函数的性质
    \begin{itemize}
      \item 有界性、最值存在性、介值定理
      \item 重点:用介值性证明解的存在性
    \end{itemize}
  \end{itemize}
  \item 无穷小代换(重点+熟练掌握)
  \begin{itemize}
    \item 熟练掌握$x\to 0$时,常用的等价无穷小
    \item 了解无穷小相关的运算规则,例如:$x^m\cdot\circ(x^n)=\circ(x^{n+m})$
    \item 重点掌握用无穷小代换计算极限:几乎可用于任何极限计算问题
    \begin{itemize}
      \item 原则:只能代换乘法因子!
      \item 计算过程中注意与变量替换的结合
    \end{itemize}
  \end{itemize}
\end{enumerate}

\bigskip

{\bf 五、导数与微分}
\begin{enumerate}
  \item 导数
  \begin{itemize}
    \item 熟悉各种记号:$f'(x_0), y'(x_0),y'_x|_{x=x_0},
    \left.\df{\d y}{\d x}\right|_{x=x_0},
    \left.\df{\d f(x)}{\d x}\right|_{x=x_0},\;\df{\d}{\d x}y,\;\df{\d}{\d
    x}f(x)$
    \item 高阶导数:$y''_{xx},\;y^{(n)},\;\df{\d^2y}{\d x^2},\;\df{\d^{n}y}{\d
    x^n},\;\df{\d^{(n)}}{\d x}y$
  \end{itemize}
  \item 导数的计算(重点+熟练掌握)
  \begin{itemize}
    \item 复合函数求导:链式法则:$[f(g(x))]'_x=f'(g(x))g'(x)$
    \begin{itemize}
      \item 注意$f'[g(x)]$与$[f(g(x))]'$的区别
      \item 参考求导测试(50分钟50题)强化公式记忆
    \end{itemize}
    \item 反函数求导:掌握公式推导方法(习题4.2-20),知道如何利用公式验证
    $e^x$与$\ln x$,以及$\sin x$与$\arcsin x$的导数的关系
    \item 参数方程求导(难点):设$x=\varphi(t),y=\psi(t)$,则
    \begin{itemize}
      \item $y'_x=\df{\d y}{\d x}=\df{\d y}{\d t}\df{\d t}{\d x}
      =\df{y'_t(t)}{x'_t(t)}=\df{\psi'(t)}{\varphi'(t)}$
      \item
      $y''_{xx}=\df{\d y'_x}{\d x}=\df{\d y'_x}{\d t}\df{\d t}{\d x}
      =\df{(y'_x)'_t}{x'_t(t)}
%       =\df{\left[\df{\psi'(t)}{\varphi'(t)}\right]'_t}{\varphi'(t)}
      =\df{\psi''(t)\varphi'(t)-\psi'(t)\varphi''(t)}{[\varphi'(t)]^3}$
    \end{itemize}
    \item 形如$x^x,x^{x^x},\sin x^{\tan x},f(x)^{g(x)}$的导数
    $$\left[f(x)^{g(x)}\right]'=\left[e^{g(x)\ln f(x)}\right]'
    =e^{g(x)\ln f(x)}\left[g'(x)\ln f(x)+g(x)\cdot\df{f'(x)}{f(x)}\right]$$
  \end{itemize}
  \item 微分
  \begin{itemize}
    \item 正确掌握微分的写法!例如:$y=e^x$在$x=1$的微分:由于$(e^x)'_{x=1}=e$,故
    $$\left.\d y\right|_{x=1}=e\d x$$
    \item 在微分的表达式中,$x$和$\d x$是两个无关的量!
  \end{itemize}
  \item 导数的应用(重点+熟练掌握)
  \begin{itemize}
    \item 曲线的切线、法线
    \item 极值与最值,重点是极值、拐点的判定条件
    \item 函数的单调性、曲线的凹凸性、渐进线\ps{求渐近线时注意要使用单侧极限}
    \begin{itemize}
      \item 用单调性证明不等式(重点):一阶导数无法比较时,利用更高阶导数进行比较
    \end{itemize} 
	\item 弧微分:掌握各种表达式及其推导方法
	$$\d s=\sqrt{1+(y')^2}\d x=\sqrt{(x'_t)^2+(y'_t)^2}\d t
	=\sqrt{\rho^2+(\rho')^2}\d\theta$$
    \item 曲率:$K=\df{|y''|}{[1+(y')^2]^{3/2}}=
    \df{|x'_ty''_{tt}-x''_{tt}y'_t|}
	{\{[x'_t]^2+[y'_t]^2\}^{3/2}}$
%     \item L'Hospital法则:简单易用,但要注意在有些情况下不能使用({习题5.2-15})
  \end{itemize}
\end{enumerate}

\bigskip

{\bf 六、微分中值定理与Taylor公式({重点+难点})}
\begin{enumerate}
  \item 中值定理
  \begin{itemize}
    \item 为什么Rolle定理的三个条件缺一不可?
    \item 掌握Lagrange中值定理的证明(体会辅助函数的构造方法)
    \item 解题时,通常要从结果出发,考虑辅助函数的构造,例如:
    \begin{itemize}
      \item $y+\lambda y'=0$:令$F(x)=e^{\lambda x}y$
	  \item $ny+xy'=0$:令$F(x)=x^ny$
	  \item $f'(\xi)g(\xi)+f(\xi)g'(\xi)=0$:令$F(x)=f(x)g(x)$
	  \item $f'(\xi)g(\xi)-f(\xi)g'(\xi)=0$:令$F(x)=\df{f(x)}{g(x)}$
    \end{itemize}
  \end{itemize}
  \item Taylor公式=Taylor多项式+余项
  \begin{itemize}
    \item Peano余项:多用于极限的计算(余项的阶数与Taylor公式的阶数相对应
    \item Lagrange余项:多用于证明题和不等式估值
    \item 熟练掌握一些常用的Maclaurin公式:
    $e^x,\;\sin x,\;\cos x,\;\ln(1+x),\;\df{1}{1+x}$
    \item Taylor展开的方法
    \begin{itemize}
      \item 利用变量替换化成常用形式,再套用公式
      \item 多项式函数的Taylor展开式是其在展开点处对应的截断多项式
      \item 注意余项的正确写法!
    \end{itemize}
  \end{itemize}
  \item 用Taylor公式证明等式/不等式(难点)
  \begin{itemize}
    \item 使用带Lagrange余项的展开
	\item 展开过程中可能用到的点:区间端点、极(最)值点($f'(x)=0$)、
	  区间中点(到两端点距离相同)、相距常数距离的点(如$x,x+1$)、已知条件中提到的点
  \end{itemize}
  \item 用Taylor展开计算极限(易错,不推荐使用)
  \begin{itemize}
      \item 通过试探性展开确定合适的展开阶数
      \item 及时将高阶无穷小进行合并
      \item 注意与无穷小代换、L'Hospital法则等结合使用
      \item 计算过程中及时化简
    \end{itemize}
\end{enumerate}

\bigskip

{\bf 七、不定积分与定积分(重点)}
\begin{enumerate}
  \item 不定积分、原函数与变限积分
  \begin{itemize}
    \item 掌握变限积分求导公式的推导方法
    $$\left[\dint_{\varphi(x)}^{\psi(x)}f(t)\d t\right]'_x
	=f[\varphi(x)]\varphi'(x)-f[\psi(x)]\psi'(x)$$
	\item 注意:
	\begin{itemize}
	  \item 利用以上公式对变限积分求导时,被积式中不能包含$x$
	  \item 若$x$出现在被积函数中,应通过适当变换将其移到积分符号外或积分限上,要求
	  掌握对$\dint_a^x(t-x)f(t)\d t$和$\dint_a^xtf(t-x)\d t$求导的方法
	\end{itemize}
  \end{itemize}
  \item 不定积分的计算
  \begin{itemize}
    \item 换元的原则:优先替换“最不顺眼”(不易进行运算、推导)的部分,例如:
    $\sqrt{1-x},\sqrt{1+x^3},\arctan x,\ldots$
    \item 对照教材上的例题熟练各种积分换元方法
    \item 积分表(见同济大学教材附录):要求$90\%$以上达到熟练
	\item 分部积分法,熟练掌握形如:$\dint x\sin x\d x$,
	$\dint e^x\sin x\d x$、$\dint x\ln x\d x$的积分
	\item 有理函数积分(熟练掌握),步骤
	\begin{itemize}
	  \item 首先,利用多项式除法将假分式化为多项式和真分式的和
	  \item 利用待定系数法对真分式进行分解
	  \item 对分解后得到的分式逐个进行积分
	\end{itemize}
  \end{itemize}
  \item 定积分
  \begin{itemize}
    \item 掌握用定积分的定义计算极限的方法(参见“求极限的常用方法及典型例题”)
    \item 定积分的计算,注意换元后必须修改积分的上下限!
    \item 熟练掌握定积分计算的一些特殊方法
  	\begin{itemize} 
	  \item 利用对称性计算定积分
	  \item 周期函数的定积分 
	  \item 利用几何意义计算定积分
	\end{itemize}
  \end{itemize}
  \item 定积分的应用:微元法
  \begin{itemize} 
	\item 画图,在图中标出各种相关的量和符号,例如:$\d S,\d x,x,f(x)$
	\item 写出微元表达式,给出对应的积分变量变化范围,例如:
	  $\d S=f(x)\d x,x\in[a,b]$
	\item 计算积分:$\dint_a^bf(x)\d x$
  \end{itemize}
  \item 须重点掌握的应用:平面图形的面积(包括极坐标下的)、旋转体的体积、已知
  截面的立体体积、变力做功和水压问题
  \item 反常积分
  \begin{itemize}
    \item 收敛的判定方法与无穷级数收敛的判定法加以对照记忆
  \end{itemize}
\end{enumerate}

\bigskip

{\bf 附录}
\begin{enumerate}
  \item 常用不等式
  \begin{itemize}
    \item 绝对值不等式:$||a|-|b||\leq |a-b|\leq |a|+|b|$
    \item 平均值不等式:$\sqrt[n]{\prod\limits_{k=1}^n}a_k
    \leq\df 1n{\sum\limits_{k=1}^n}a_k
    \quad (a_k\geq 0)$
%     \item Bernoouli不等式:$(1+h)^n\geq 1+nh,\;(h\geq -2,\,n\in\mathbb{N})$
	\item Cauchy不等式:$\left(\sum\limits_{i=1}^na_ib_i\right)\leq
	\sum\limits_{i=1}^na_i^2\sum\limits_{i=1}^nb_i^2$
	\item Schwarz积分不等式:$\left[\dint_a^bf(x)g(x)\d x\right]^2
	\leq\dint_a^bf\,^2(x)\d x\dint_a^bg^2(x)\d x$ 
  \end{itemize}
  \item 常用处理技巧
  \begin{itemize}
    \item $a^n-b^n=(a-b)(a^{n-1}+a^{n-2}b+\ldots+ab^{n-2}+b^{n-1})$及其
    常见的变形
    \begin{itemize}
      \item $a^n-1=(a-1)(a^{n-1}+a^{n-2}+\ldots+a+1)$
      \item $\sqrt[n]a-\sqrt[n]b=\df{a-b}{\sqrt[n]{a^{n-1}}
      +\sqrt[n]{a^{n-2}b}+\ldots+\sqrt[n]{ab^{n-2}}+\sqrt[n]{b^{n-1}}}$
	\end{itemize}
	\item “搭桥”(通过“加一项减一项”的方式使公式易于推导),例如:
      \begin{align}
      	\limx{0}\df{1-\cos2x\cos x}{x^2}&=\limx0\df{(1-\cos x)+
      	\cos x(1-\cos2x)}{x^2}\notag\\
      	&=\limx0\df{1-\cos x}{x^2}+\limx0\cos x
      \cdot\limx0\df{1-\cos2x}{x^2}=\ldots\notag
      \end{align}
    \item 乘法的“搭桥”,例如:
    $$\cos\df x2\cos\df x{4}\ldots\cos\df x{2^k}
    =\df{\cos\df x2\cos\df x{4}\ldots\cos\df x{2^k}\sin\df x{2^k}}
    {\sin\df x{2^k}}=\ldots=\df{\sin x}{2^k\sin\df x{2^k}}$$
    又比如:参数方程求导时,已知$x=x(t),y=y(t)$,则
    $$\df{\d y}{\d x}=\df{\d y}{\d t}\cdot\df{\d t}{\d x}=\df{y'_t}{x'_t}$$
    \item “夹逼”,例如:计算极限$\limn\sqrt[n]a\;(a>0)$(P61-例6);又例如
    应用中值定理计算极限的处理技巧(参见“求极限的常用方法及典型例题”)
    \item 有理分式的分解,例如:设$a,b,c$均不相等
    $$\df1{(x-a)(x-b)}=\df1{a-b}\cdot\left(\df1{x-a}-\df1{x-b}\right)$$
    \begin{align}
    	\df1{(x-a)(x-b)(x-c)}&=\df1{a-b}\cdot\left[\left(\df1{x-a}-\df1{x-b}\right)
    	\cdot\df1{x-c}\right]\notag\\
    	&\hspace{-2cm}=\df1{a-b}\left[\df1{a-c}\cdot\left(\df1{x-a}-\df1{x-c}
    	\right)-\df1{b-c}\cdot\left(\df1{x-b}-\df1{x-c}\right)\right]\notag
    \end{align}
  \end{itemize}
  \item 形如$\sin(\arctan x)$的推导方法,举例如下:
  \begin{itemize}
    \item $\sin(\arccos x)=\sqrt{1-\cos^2(\arccos x)}=\sqrt{1-x^2}$
    \item 注意到$\sin x=\tan x\cos x=\df{\tan x}{\sec x}=\df{\tan x}
    {\sqrt{1+\tan^2x}}$,故
    $$\sin(\arctan x)=\df{\tan(\arctan x)}{\sqrt{1+\tan^2(\arctan x)}}
    =\df{x}{\sqrt{1+x^2}}$$
    \item 注意到$\tan x=\df{\sin x}{\cos x}=\df{\sin x}{\sqrt{1-\sin^2x}}$,故
    $$\tan(\arcsin x)=\df{\sin(\arcsin x)}{\sqrt{1-\sin^2(\arcsin x)}}
    =\df{x}{\sqrt{1-x^2}}$$
  \end{itemize}
  \item 形如$a\sin\theta+b\cos\theta\,(a^2+b^2\ne 0)$的处理方法:
  \begin{itemize}
    \item 设$\tan\phi=\df ba$,则$\cos\phi=\df a{\sqrt{a^2+b^2}},
    \sin\phi=\df b{\sqrt{a^2+b^2}}$,从而
    $$a\sin\theta+b\cos\theta=\sqrt{a^2+b^2}
    (\cos\phi\sin\theta+\sin\phi\cos\theta)=\sqrt{a^2+b^2}\sin(\theta+\phi)$$
    \item 类似地,若令$\tan\phi=\df ab$,则$\cos\phi=\df b{\sqrt{a^2+b^2}},
    \sin\phi=\df a{\sqrt{a^2+b^2}}$,从而
    $$a\sin\theta+b\cos\theta=\sqrt{a^2+b^2}
    (\sin\phi\sin\theta+\cos\phi\cos\theta)=\sqrt{a^2+b^2}\cos(\theta-\phi)$$
    \item 应用举例:令$\tan\phi=\df ba$
    \begin{align}
    	\dint&\df{\sin x}{a\sin x+b\cos x}\d x
    	=\dint\df{\sin x}{\sqrt{a^2+b^2}\sin(x+\phi)}\d x\notag\\
    	&=\dint\df{\sin [(x+\phi)-\phi]}{\sqrt{a^2+b^2}\sin(x+\phi)}\d (x+\phi)
    	\quad(\mbox{令}t=x+\phi)\notag\\
    	&=\df1{\sqrt{a^2+b^2}}\dint\df{\sin(t-\phi)}{\sin t}\d t
    	=\df1{\sqrt{a^2+b^2}}\dint\df{\sin t\cos\phi-\cos t\sin\phi}
    	{\sin t}\d t\notag\\
    	&=\ldots\quad(\mbox{再往下会算了吧:P})\notag
    \end{align}
    \item 同样的方法,可以计算所有形如$\dint R(\sin x,\cos x,a\sin x+b\cos x)\d x$
    的积分
  \end{itemize}
  
\end{enumerate}
