\begin{center}
	{\Large\bf 单元测验:多元函数微分学、微分方程与空间解析几何}
	
	(时间:150分钟)
\end{center}

{\bf 一、填空题(每题3分)}

1.\;设$x,e^x,e^{2x}$分别为方程$y''+p(x)y'+q(x)y=f(x)$的三个解,则
该方程满足

\bigskip

\hspace{1em}$y(0)=1,y'(0)=3$的特解为\underline{\hspace{4cm}}
% $2e^{2x}-e^x$

\bigskip

2.\;设$F(x,y)=\dint_0^{xy}\df{\sin t}{1+t^2}\d t$,则
$\left.\df{\p^2F}{\p x^2}\right|_{x=0 \atop y=2}=$\underline{\hspace{4cm}}
% 4

\bigskip

3.\;设$z=f\left(xy,y\sin\df{\pi x}2\right)$,其中$f$二阶偏导数连续,
则$z''_{xy}(1,1)=$\underline{\hspace{4cm}}
% $f''_{11}(1,1)+f''_{12}(1,1)+f'_1(1,1)$

\bigskip

4.\;以$y=7e^{3x}+2x$为特解的三阶常系数齐次线性微分方程为\underline{\hspace{4cm}}
% $y'''-3y''=0$

\bigskip

5.\;设$f(x,y)=\left\{\begin{array}{ll}
	\df1{xy}\sin(x^2y),& xy\ne 0\\
	0,& xy=0
\end{array}\right.$,则$f'_x(0,1)=$\underline{\hspace{4cm}}
% 1

\bigskip

{\bf 二、选择题(每题3分)}

1.\;方程$y''+y=x^2+1+\sin x$的特解可设为
  (\underline{\hspace{1cm}})
  %\ps{C}
  \begin{enumerate}[(A)]
  \setlength{\itemindent}{3em}
    \item $y^*=ax^2+bx+c+x(A\sin x+B\cos x)$
    \item $y^*=x(ax^2+bx+c+A\sin x+B\cos x)$
    \item $y^*=ax^2+bx+c+A\sin x$
    \item $y^*=ax^2+bx+c+A\cos x$
  \end{enumerate}

\bigskip
 
2.\;二重极限$\lim\limits_{(x,y)\to(0,0)}\df{xy^2}{x^2+y^4}=$
  (\underline{\hspace{1cm}})
  %\ps{D}  
  
  \hspace{2ex}(A) $0$\hspace{1cm}(B) $1$ 
  \hspace{1cm}(C)$\df12$\hspace{1cm}(D)不存在
  
\bigskip

3.\;设$f(x,y)=(y-x^2)(y-x^4)$,$P(0,0),M(1,1)$,则(\underline{\hspace{1cm}})
%   \ps{B}
  \begin{enumerate}[(A)]
  \setlength{\itemindent}{3em}
    \item $P,M$均为$f(x,y)$的极值点
    \item $P,M$均不是$f(x,y)$的极值点
    \item $P$为$f(x,y)$的极值点,$M$不是
    \item $M$为$f(x,y)$的极值点,$P$不是
  \end{enumerate}

\bigskip

4.\;$x^2+y^2=1$时,$f(x,y)=(x^2+y^2)e^{-(x^2+y^2)}$ 
  (\underline{\hspace{1cm}})
%   \ps{B}
  
  \hspace{2ex}(A)不取极值\hspace{1cm}(B)取极大值 \hspace{1cm}
  (C)取极小值\hspace{1cm}(D)取最大值

\bigskip

5.\;$f(x,y)$在原点附近连续,$\lim\limits_{(x,y)\to(0,0)}
  \df{f(x,y)-|xy|}{(x^2+y^2)^2}=1$,则$f(x,y)$在原点
  (\underline{\hspace{1cm}})
%   \ps{C}
  
  \hspace{2ex}(A)不取极值\hspace{1cm}(B)取极大值 \hspace{1cm}(C)取极小值\hspace{1cm}
  (D)不一定取极值

\bigskip

{\bf 三、}(6分)求函数$z=1-x^2-y^2$在点$(1,1)$处,沿曲线$x^2+y^2=2$的内
法线方向的方向导数。

\bigskip

{\bf 四、}(6分)设$u=u(x)$由
$$u=f(x,y),\quad g(x,y,z)=0,\quad h(x,z)=0$$
确定,$f,g,h$一阶偏导连续,$h'_z\ne0,g'_y\ne0$,求$\df{\p u}{\p x}$

\bigskip

{\bf 五、}(6分)设$f(x,y)=\dint_0^{xy}e^{-t^2}\d t$,求
$$\df xy\df{\p^2f}{\p x^2}
-2\df{\p^2f}{\p x\p y}+\df yx\df{\p^2f}{\p y^2}$$

\bigskip

{\bf 六、}(6分)已知$z=f(x,y)$在$(x_0,y_0)$处可微,证明:$f(x,y)$在$(x_0,y_0)$
处沿任意方向的方向导数存在。

\bigskip

{\bf 七、}(6分)已知曲面方程$\Sigma:z=xe^{y/x}$,证明:$\Sigma$上任一点$M$处的法线与其
向径垂直。

\bigskip

{\bf 八、}(8分)设$f(x,y)$在原点连续,且
$$\lim\limits_{(x,y)\to(0,0)}\df{f(x,y)-a-bx-cy}{\ln(1+x^2+y^2)}=1,$$
其中$a,b,c$为常数。
\begin{enumerate}[(1)]
  \setlength{\itemindent}{1cm}
  \item 讨论$f(x,y)$在原点是否可微,若可微,求出$\left.\d f(x,y)\right|_{(0,0)}$;
  \item 讨论$f(x,y)$在原点是否取极值,说明理由。
\end{enumerate}

\bigskip

{\bf 九、}(8分)设$f(u)$但$u>0$时二阶可导,且$z=f(\sqrt{x^2+y^2})$满足
$z''_{xx}+z''_{yy}=0$。
\begin{enumerate}[(1)]
  \setlength{\itemindent}{1cm}
  \item 验证:$uf''(u)+f'(u)=0$;
  \item 若$f(1)=0,\;f'(1)=1$,求$f(u)$。
\end{enumerate}

\bigskip

{\bf 十、}(8分)求$z=x^2y(4-x-y)$在由直线$x+y=6$、$x$轴和$y$轴所围闭区域$D$内的
最大值、最小值和所有极值。

\bigskip

{\bf 十一、}(8分)设$A,B,C$为常数,$B^2-AC>0,\;A\ne 0$,$u(x,y)$二阶偏导函数均连续,
证明:存在非奇异线性变换
$$\xi=ax+y,\quad \eta=bx+y\quad(a,b\mbox{为常数})$$
将方程$A\df{\p^2 u}{\p x^2}+2B\df{\p^2 u}{\p x\p y}+C\df{\p^2 u}{\p y^2}=0$
化为$\df{\p^2u}{\p\xi\p\eta}=0$。

\bigskip

{\bf 十二、}(8分)设$f(u)$连续可导,$f(2)=1$,且函数$z=xf\left(\df yx\right)
+yf\left(\df yx\right)$满足
$$\df{\p z}{\p x}+\df{\p z}{\p y}=\df yx-\left(\df yx\right)^3,\;x>0,y>0.$$
求$f(u)$。

\newpage

\begin{center}
	{\Large\bf 解答与评分标准}\ps{\b 
	1.若解法与参考答案不同,参照本评分标准的特点分段给分\\
	2.计算题只写结果,缺少计算过程,最多得1分}
\end{center}

{\bf 一、填空题(每题3分)}

1.\;$2e^{2x}-e^x$\quad\quad2.\;$4$\quad\quad
3.\;$f''_{11}(1,1)+f''_{12}(1,1)+f'_1(1,1)$\quad\quad4.\;$y'''-3y''=0$
\quad\quad5.\;$1$

{\bf 二、选择题(每题3分):}

\quad C\quad D\quad B\quad B\quad C

{\bf 三、}解:
$$z'_x(1,1)=-2,\quad z'_y(1,1)=-2,\eqno{(+2\mbox{分})}$$
又$\bigtriangledown(x^2+y^2)|_{(1,1)}=(2,2)$,且$f(x,y)=x^2+y^2$
沿着外法向递增,故其内法向的为$\bm{u}=(-2,-2)$,其对应的单位向量为
$\left(-\df{\sqrt2}2,-\df{\sqrt2}2\right)$
\hfill{($+2$分)}

综上,所求方向导数
$$D_uz(1,1)=z'_x(1,1)\left(-\df{\sqrt2}2\right)
+z'_x(1,1)\left(-\df{\sqrt2}2\right)=2\sqrt2\eqno{(+2\mbox{分})}$$

{\bf 四、}解:\ps{本题用已知方程两边求微分的方法求解亦可}
已知三个方程中包含四个变量,故若以$x$为自变量,则$u,y,z$均应视为
$x$的函数,于是,对已知的三个方程两边同时关于$x$求导,可得
$$
	\left\{\begin{array}{l}
		u'_x=f'_x+f'_yy'_x0\\
		g'_x+g'_yy'_x+g'_zz'_x=0\\
		h'_x+h'_zz'_x=0
	\end{array}\right.\eqno{(+4\mbox{分})}
$$
由此可解得
$$u'_x=f'_x+f'_y\df{g'_zh'_x-g'_xh'_z}{g'_yh'_z}\eqno{(+2\mbox{分})}$$

{\bf 五、}答案:$-2e^{-x^2y^2}$\ps{过程略,根据过程酌情给分}

{\bf 六、}参考教材P118-定理10.4.1的证明。

{\bf 七、}证:设$F=z-xe^{y/x}$,则
$$\bigtriangledown F=\left(-e^{y/x}+\df yxe^{y/x},-e^{y/x},1\right),$$
即为曲面$\Sigma$的法向量。\hfill{($+4$分)}

$\Sigma$上任一点$M$处的向径$\bm{r}=\left(x,y,z\right)
=\left(x,y,xe^{y/x}\right)$,故
$$\bigtriangledown F\cdot\bm{r}=\left(-e^{y/x}+\df yxe^{y/x}\right)x
-ye^{y/x}+xe^{y/x}=0\eqno{(+2\mbox{分})}$$
即证。

{\bf 八、}解:(1)$(x,y)\to(0,0)$时,$\ln(1+x^2+y^2)\sim x^2+y^2$,故
已知极限即为
$$\lim\limits_{(x,y)\to(0,0)}\df{f(x,y)-a-bx-cy}{x^2+y^2}=1,$$
从而可得
$$\lim\limits_{(x,y)\to(0,0)}[f(x,y)-a-bx-cy]=0\quad\Rightarrow\quad
\lim\limits_{(x,y)\to(0,0)}f(x,y)=a.$$
因为$f(x,y)$在原点连续,故$f(0,0)=a$。
\hfill{($+3$分)}

于是已知极限可改写为
$$f(x,y)-f(0,0)=bx-cy+\circ(\sqrt{x^2+y^2}),$$
由可微的定义,可知$f(x,y)$在原点处可微。\hfill{($+1$分)}

(2)由(1)可知
$$\d f(x,y)|_{(0,0)}=b\d x+c\d y,$$
即$\bigtriangledown f(0,0)=(b,c)$。由可微函数极值的必要条件($\bigtriangledown f=0$),
当$b,c$不全为零时,原点不是$f(x,y)$的极值点。\hfill{($+2$分)}

当$b=c=0$时,由于
$$\lim\limits_{(x,y)\to(0,0)}\df{f(x,y)-f(0,0)}{x^2+y^2}=1>0,$$
由极限的保号性,必存在原点的某个去心领域$U$,使对任意$(x,y)\in U$,有
$$\df{f(x,y)-f(0,0)}{x^2+y^2}>0\quad\Rightarrow\quad
f(x,y)>f(0,0),$$
由此可知原点为$f(x,y)$的极小值点。\hfill{($+2$分)}

{\bf 九、}解:(1)略。

(2)$f(u)=\ln u$,过程略。

{\bf 十、}解:令$\bigtriangledown z=0$,可解得驻点
$$x=0(0\leq y\leq 6),\quad (4,0),\quad (2,1)\eqno{(+2\mbox{分})}$$
其中只有$(2,1)$为区域$D$的内点,故只有其可能为极值点。

在点$(2,1)$处,$\bigtriangledown^2 z=\left[\begin{array}{cc}
	-6 & -4\\ -4 & -8
\end{array}\right]$为负定矩阵,故$z$此时取极大值$z(1,2)=4$.\hfill{($+2$分)}

以下讨论区域$D$的三条边界上的情形:

在$x=0,0\leq y\leq 6$和$y=0,0\leq x\leq 6$上,$f(x,y)=0$;

在$x+y=6$上,$y=6-x$,从而$z=2x^3-12x^2\;(0\leq x\leq 6)$,其最大和最小值分别为
$z(0,y)=z(x,0)=0$和$z(4,2)=-64$。\hfill{($+4$分)}

综上,在区域$D$上,$z$的最大和最小值分别为$z(2,1)=4$和$z(4,2)=-64$。

{\bf 十一、}证:依题意,
$$\df{\p^2 u}{\p x^2}=a^2\df{\p^2 u}{\p \xi^2}+2ab\df{\p^2 u}{\p\xi\p\eta}
+b^2\df{\p^2 u}{\p\eta^2}$$
$$\df{\p^2 u}{\p y^2}=\df{\p^2 u}{\p \xi^2}+2\df{\p^2 u}{\p\xi\p\eta}
+\df{\p^2 u}{\p\eta^2}$$
$$\df{\p^2 u}{\p x\p y}=a\df{\p^2 u}{\p \xi^2}+(a+b)\df{\p^2 u}{\p\xi\p\eta}
+b\df{\p^2 u}{\p\eta^2}\eqno{(+4\mbox{分})}$$
代入已知方程可得
$$(Aa^2+2Ba+C)\df{\p^2 u}{\p \xi^2}+2(Aab+B(a+b)+C)\df{\p^2 u}{\p\xi\p\eta}
+(Ab^2+2Bb+C)\df{\p^2 u}{\p\eta^2}=0$$

由于$B^2-AC>0,\;A\ne0$,故方程$Ax^2+2Bx+C=0$必有相异实根$x_1,x_2$。令$a=x_1,b=x_2$,
则$a+b=-\df{2B}A,\;ab=\df CA$,进而
$Aab+B(a+b)+C=\df2A(AC-B^2)\ne 0$
故上式可化为$\df{\p^2 u}{\p\xi\p\eta}=0$,即证。\hfill{($+4$分)}

{\bf 十二、}解:将$z=xf\left(\df yx\right)+yf\left(\df yx\right)$代入已知方程,
并令$u=\df yx$,化简可得
$$f'(u)(1-u^2)+2f(u)=u-u^3,\;(u>0)\eqno{(+4\mbox{分})}$$

当$u\ne 1$时,上式即为
$$f'(u)+\df2{1-u^2}f(u)=u,$$
这是一个一阶非齐次线性微分方程,解之可得
$$f(u)=\df{u-1}{u+1}\left(\df{u^2}2+2u+2\ln|u-1|-3\right)
\;(u>0)\eqno{(+4\mbox{分})}$$