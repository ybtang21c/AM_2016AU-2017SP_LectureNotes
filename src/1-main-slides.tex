% !Mode:: "TeX:UTF-8"

%Composed by Y.B. TANG (ybtang21c@gmail.com), spring-2011
%tex distribution: Texlive / MikTex 2010
%recommended editer: Eclipse + Texlipse
%usage: compile with XeLatex

%option: red, brown, blue
% \documentclass[14pt,mathserif]{beamer}
% \documentclass[14pt]{beamer}
\documentclass[usepdftitle=false, 14pt]{beamer}

\usepackage{amsmath,amsfonts,amssymb,amsthm,bm}
% \usepackage{txfonts} %another style of math fonts
\usepackage{beamerthemesplit,color,graphics}
\usepackage{ulem} %erase line
\usepackage{esint} %any type of integral symbol
% \usepackage{yhmath} %圆弧帽:\wideparen{AB}

\usepackage{bbding}%各种五角星
% \FiveStar,\FiveStarOpen,\FiveStarLines,\FiveStarShadow
% \FiveStarOutline,\FiveStarCenterOpen,\FiveStarOpenDotted
% \FiveStarConvex,\FiveStarOutlineHeavy,\FiveStarOpenCircled

%==============XeCJK==============================
\usepackage[slantfont,boldfont,CJKchecksingle]{xeCJK}
\setmainfont{Times New Roman}
\setCJKmainfont[BoldFont={Adobe Heiti Std},
	ItalicFont={Adobe Kaiti Std},
    SlantedFont={Adobe Song Std},
%     BoldItalicFont={Weibei SC},
     BoldSlantedFont={Adobe Fangsong Std}
	]{Adobe Heiti Std}
\punctstyle{CCT}
\usepackage{xeCJKfntef}%汉字加点和可断行的下划线
% 
% \newCJKfontfamily[Wawa]\Wawati{Wawati SC}
% \newCJKfontfamily[STLi]\stliti{STLiti}

\usefonttheme{professionalfonts}

\newcommand{\fs}[1]{\fontspec{#1}\CJKfontspec{#1}}

%

% %==============std fontspec settings==============
% \usepackage[no-math,cm-default]{fontspec}
% % \newfontfamily\zhfont[BoldFont=Adobe Heiti Std]{Adobe Heiti Std}
% \newfontfamily\zhfont[BoldFont=Adobe Heiti Std]{Adobe Kaiti Std}
% % 
% % %==============spacing of CH in Xetex==============
% \usepackage{zhspacing}
% \zhspacing

%==============layout setting==============
\setlength{\parindent}{0pt}  

%==============beamer configuration==============
\setbeamertemplate{theorems}[normal font]

%============beamer theme setting #2============
\mode<presentation>{
	\usetheme{Copenhagen} %Copenhagen, Warsaw, CambridgeUS
  	\usecolortheme{seahorse} %rose, seahorse, lily, crane
  	\usefonttheme{serif}
  	\usefonttheme{structurebold}
  	\useoutertheme{infolines}
%   \beamertemplateshadingbackground{brown!5}{yellow!10}
% 	\setbeamercolor{frametitle}{bg=white}	
%  	\setbeamercovered{transparent}
}

%============TOC setting============
% \AtBeginSection{
%    \begin{frame}{内容提要}
%      \tableofcontents[currentsection,hideallsubsections]
%    \end{frame}
% }
% \AtBeginSubsection{
%    \begin{frame}{内容提要}
%      \tableofcontents[currentsection,currentsubsection]
%    \end{frame}
% }
% \AtBeginSection{
%   \frame{\tableofcontents[sections={\thesection}]}
% }

%===============macros====================
\newcommand{\bb}{\bf\color{blue}}
\newcommand{\ba}[1]{\alert{\bf #1}}
\newcommand*{\e}{\ensuremath{\varepsilon}}
\renewcommand{\b}{\color{blue}}
\newcommand*{\p}{\ensuremath{\partial}}
\newcommand{\limn}{\ensuremath{\lim\limits_{n\to\infty}}}
\newcommand{\sumn}{\ensuremath{\sum\limits_{n=1}^{\infty}}}
\newcommand*{\df}[2]{\displaystyle\frac{\,{#1}\,}{\,{#2}\,}}
\newcommand*{\limx}[1]{\ensuremath{\lim\limits_{x\to{#1}}}}
\newcommand*{\limdx}{\ensuremath{\lim\limits_{\Delta x\to 0}}}
\newcommand*{\dx}{\Delta x}
\newcommand{\dint}{\ensuremath{\displaystyle\int}}
\renewcommand{\d}{\mathrm{d}}
\newcommand{\ds}{\displaystyle}

%================exampleblock counter===================
% \newcounter{examplecounter}
% \usecounter{examplecounter}
% \setcounter{examplecounter}{1}
% \newcommand{\exno}{{\bf
% 例\arabic{examplecounter}}\refstepcounter{examplecounter}}

%================block setting test=====================
% \definecolor{beamer@blendedred}{rgb}{0.7,0.2,0.2} % use structure theme to change
% \definecolor{beamer@blendedblue}{rgb}{0.2,0.2,0.7} % use structure theme to change
% \definecolor{beamer@blendedyellow}{rgb}{0.7,0.7,0.2}

% \setbeamercolor{structure}{fg=beamer@blendedred}

% \setbeamercolor{block title}
% {use=structure,fg=structure.fg,bg=structure.fg!20!bg}
% \setbeamercolor{block title alerted}
% {use=alerted text,fg=alerted text.fg,bg=alerted text.fg!20!bg}
% \setbeamercolor{block title example}
% {use=example text,fg=example text.fg,bg=example text.fg!20!bg}

% \setbeamercolor{block body}
% {parent=normal text,use=block title,bg=block title.bg!50!bg}
% \setbeamercolor{block body alerted}
% {parent=normal text,use=block title alerted,bg=block title alerted.bg!50!bg}
% \setbeamercolor{block body example}
% {parent=normal text,use=block title example,bg=block title example.bg!50!bg}

% \setbeamercolor{titlelike}{parent=structure,bg=white!90!red}
%=======================================================

%===============title setting====================

\title{Advanced Mathematics II}
\author[NUDT]{唐扬斌\\
\texttt{\small ybtang21c@gmail.com\\tangyangbin@gfkd.mtn\\CP: 18374857376}}
\institute[NUDT]{National University of Defense Technology}
\date[Spring 2016]{Spring 2016}

%===============document begins here==============

\begin{document}

%L-1.tex:2010秋季学期考试讲评
%L01.tex:常微分方程的概念与一阶微分方程的解法
%L04-Ch07-1orderDE-HW:一阶微分方程求解的课堂思考题
%L02.tex:二阶微分方程的解法
%L03_ch7ex1.tex:常微分方程习题课
%L04.tex:向量运算与空间直线、平面的方程
%L05.tex:空间平面与直线
%L05-Ch08-ex1.tex
%L06.tex:空间曲面
%L07.tex:空间曲线
%L08_ch8ex2.tex:空间曲面与曲线习题课
%L09.tex:向量值函数
%L10.tex:多元函数极限、连续与偏导数
%L11.tex:全微分与复合函数求导
%L12.tex:多元复合函数与隐函数的偏导数
%L13_ch9ex1:多元函数的偏导数习题课
%L14.tex:方向导数与梯度、Hessian矩阵及Taylor公式
%L15.tex:多元函数的极值与条件极值
%L16_ch10ex2.tex:多元函数微分的应用
%L16_ch10ex3.tex:多元函数微分学及其应用(All)
%L17.tex:重积分的概念与性质
%L18.tex:重积分的计算
%L19.tex:坐标变换与重积分的计算
%L20.tex:重积分的应用
%L21.tex:重积分习题课
%L22.tex:曲线积分
%L23.tex:Green公式与保守场
%L24.tex:曲面积分
%L25.tex:Gauss公式与Stokes公式
%L26_ch12ex.tex:曲线与曲面积分习题课
% L26_ch12ex-1.tex:曲线与曲面积分习题课,2016春
%L27_ch12ex.tex:对称性在积分计算中的应用
%L28.tex:幂级数及其应用
%L29.tex:Fourier级数
%L30.tex:《高等数学(下)》复习

%fontTest.tex:各种字体的演示

%    \input{L21.tex}
   % !Mode:: "TeX:UTF-8"

\begin{frame}
	\frametitle{常微分方程习题课}
	\linespread{1.5}
	  \begin{itemize}
% 	    \item 理解微分方程及其相关概念
	    \item 一阶微分方程解法与常用解题技巧
	    \item 两类可降阶的二阶方程的解法及其高阶推广
	    \item 二阶线性微分方程解的结构
	    \item 解二阶常系数齐次线性微分方程的特征方程法及其高阶推广
	    \item 解两类二阶常系数非齐次线性微分方程的待定系数法
	  \end{itemize}
\end{frame}

\section{问题讨论}

\begin{frame}{问题讨论}
	\linespread{1.5}
	\alert{问:}已知$n$阶线性微分方程的$n$个解,
	能否写出这个微分方程及其通解?\pause\\[1ex]
	
	\alert{答:}{\it 不一定!}\pause {\it 除非 
	这$n$个解恰好线性无关。} \pause 
	
	\bigskip
	\alert{问:}适当确定微分方程通解中的参数值,可以得到其任意的特解?\pause \\[1ex]
	
	\alert{答:}{\it 错!}\pause 反例:{\it $y'=\sin x\cos^2y$}.
\end{frame}

\begin{frame}{问题讨论}
	\linespread{1.2}
	\alert{问:}$y_1=(x-1)^2$和$y_2=(x+1)^2$都是方程
	$$(x-1)^2y''-2xy'+2y=0,$$
	和
	$$2yy''-(y')^2=0$$
	的解。但二者的线性组合
	$$y=C_1(x-1)^2+C_2(x+1)^2,\;(C_1,C_2\in\mathbb{R})$$
	却仅能满足前一个方程,为什么?\pause 
	
	\alert{答:}{\it 第二个方程不是线性方程!}
\end{frame}

\section{补充例题}

\begin{frame}{填空}
	\linespread{2}
	\ba{1.}\;$y''+4y'+4y=1$的通解为
	\underline{\uncover<2->{\;\b{$(C_1+C_2x)e^{-2x}+1/4$}}\;}.\\[1em]
	
	\ba{2.}\;设$e^x(C_1\cos x+C_2\sin x)$为首项系数为$1$的某二阶常系数
	齐次线性微分方程的通解,则该微分方程为
	\underline{\uncover<3->{\;\b{$y''-2y'+2y=0$}}\;}.\\[1em]
	
	\ba{3.}\;设$\cos x$与$xe^x$分别为某$n$阶常系数齐次线性微分方程的两个解,
	则最小的$n=$\underline{\uncover<4->{\;\b{$4$}}\;},相应的首项
	系数为$1$的方程为\underline{\uncover<5->{\;\b{$
	y^{(4)}-2y^{(3)}+2y''-2y'+y=0$}\;}}
	
% 	方程$xy''-2xy'+2y=x\ln x$的通解为
% 	为\underline{\uncover<6->{\;\b{$
% 	C_1x+C_2x^2-\left(\df12\ln^2x+\ln x\right)x$}\;}}
\end{frame}

\begin{frame}{选择}
	\linespread{1.5}
	\ba{1.}\;方程$y''+4y=e^{3x}+x\sin 2x$的一个特解形式是
	(\underline{\uncover<2->{\;\b{A}}\;})
	\begin{enumerate}[(A)]
	  \item $Ae^{3x}+x[(Bx+C)\cos2x+(Dx+E)\sin2x]$
	  \item $Ae^{3x}+(Bx+C)\cos2x+(Dx+E)\sin2x$
	  \item $Axe^{3x}+x[(Bx+C)\cos2x+(Dx+E)\sin2x]$
	  \item $Axe^{3x}+(Bx+C)\cos2x+(Dx+E)\sin2x$
	\end{enumerate}
\end{frame}

\begin{frame}{选择}
	\linespread{1.3}
	\ba{2.}\;设$y_1(x),y_2(x),y_3(x)$为方程
	$$y''+p(x)y'+q(x)y=f(x)$$
	的三个线性无关的解,$C_1,C_2$为任意常数,则该非齐次线性微分方程的通解为
	(\underline{\uncover<2->{\;\b{C}}\;})
	\begin{enumerate}[(A)]
	  \item $(C_1+C_2)y_1+(C_2-C_1)y_2+(1-C_2)y_3$
	  \item $(C_1+C_2)y_1+(C_2-C_1)y_2+(C_1-C_2)y_3$
	  \item $C_1y_1+(C_2-C_1)y_2+(1-C_2)y_3$
	  \item $C_1y_1+(C_2-C_1)y_2+(C_1-C_2)y_3$
	\end{enumerate}
\end{frame}

\begin{frame}{选择}
	\linespread{1.3}
	\ba{3.}\;设$y=f(x)$为方程$y''-2y'+4y=0$
	的一个解,若$f(x_0)>0,f'(x_0)=0$,
	则函数$f(x)$在$x_0$
	(\underline{\uncover<2->{\;\b{A}}\;})
	\begin{enumerate}[(A)]
	  \item 取极大值
	  \item 取极小值
	  \item 的某个领域内单调增加
	  \item 的某个领域内单调减少
	\end{enumerate}
\end{frame}

\begin{frame}{选择}
	\linespread{1.3}
	\ba{4.}\;设$y(x)$为方程$y''+py'+qy=e^{3x}$满足初始条件$y(0)=y'(0)=0$
	的解,则$\limx{0}\df{\ln(1+x^2)}{y(x)}$
	(\underline{\uncover<2->{\;\b{C}}\;})
	\begin{enumerate}[(A)]
	  \item 不存在
	  \item 等于$1$
	  \item 等于$2$
	  \item 等于$3$
	\end{enumerate}
\end{frame}

\begin{frame}{选择}
	\linespread{1.3}
	\ba{5.}\;设$y(x)$满足$x\d y+(x-2y)\d x=0$,且曲线$y=y(x)$与直线$x=1$
	及$x$轴所围平面图形绕$x$轴旋转所得旋转体的体积最小,则$y(x)=$
	(\underline{\uncover<2->{\;\b{C}}\;})
	\begin{enumerate}[(A)]
	  \item $x-\df14x^2$
	  \item $x+\df54x^2$
	  \item $x-\df54x^2$
	  \item $x+\df14x^2$
	\end{enumerate}
\end{frame}

\begin{frame}{选择}
	\linespread{1.3}
	\ba{6.}\;方程$y''+by'+y=0$的每个解都在$x>0$上有界,则实数$b$的取值范围是
	(\underline{\uncover<2->{\;\b{A}}\;})
	\begin{enumerate}[(A)]
	  \item $[0,+\infty)$
	  \item $(-\infty,0]$
	  \item $(-\infty,4]$
	  \item $(-\infty,+\infty]$
	\end{enumerate}
\end{frame}

% \begin{frame}{解方程}
% 	\linespread{1.5}
% 	\begin{enumerate}
% 	  \item $xy'\ln x+y=\ln x$.\hfill \b$t=\ln x$
% 	\end{enumerate}
% \end{frame}

\begin{frame}{解答题}
	\linespread{1.2}
	\ba{1.}\;设$f(x)$为连续函数,且
	$$f(x)=e^{-x}+\dint_0^xf(t)\d t,$$
	求$f(x)$.
	
	\pause\alert{提示:}\it\b  积分方程通常自带初值条件$f(0)=1$.\pause
	$$f'(x)-f(x)=-e^{-x}\quad\Rightarrow\quad 
	f(x)=\df12(e^x+e^{-x}).$$
	
	\pause 不带初值条件的例子:\ba{习题7.2-8}
\end{frame}

\begin{frame}{解答题}
	\linespread{1.2}
	\ba{2.}\;设$f(x)$为连续函数,且
	$$f(x)=e^{2x}+\dint_0^xtf(x-t)\d t,$$
	求$f(x)$.
	
	\pause\alert{提示:}\it\b $f(0)=1,f'(0)=2$\pause  
	$$f''(x)-f(x)=4e^{2x}\quad\Rightarrow\quad 
	f(x)=-\df12e^x+\df16e^{-x}+\df43e^{2x}.$$
\end{frame}

% \begin{frame}{解答题}
% 	\linespread{1.2}
% 	\ba{3.(习题7.3-7)}函数$y(x)\;(x\geq 0)$二阶可导,$y'(x)>0$,
% 	$y(0)=1$,过其上任一点$(x,y)$作曲线的切线和至$x$轴的垂线,该两直线
% 	与$x$轴所围成的三角形面积记为$S_1(x)$,又区间$[0,\alert{x}]$
% 	(\alert{\it 此处教材印刷错误!})上以$y(x)$为曲边
% 	的曲边梯形面积记为$S_2(x)$。已知$2S_1-S_2=1$,求$y(x)$。
% 	
% 	\pause\alert{提示:}\it\b   
% 	$$\df{y^2}{y'}-\dint_0^xy(t)\d t=1\quad\Rightarrow\quad 
% 	yy''=(y')^2,\;y'(0)=1.$$
% 	\pause 结合$y(0)=1$,解得$y=e^x$.
% \end{frame}

\begin{frame}{解答题}
	\linespread{1.2}
	\ba{3.}\;设对任意$x,y\in\mathbb{R}$
	$$f(x+y)=f(x)e^y+f(y)e^x,$$
	$f'(0)=a\ne 0$,求$f(x)$.
	
	\pause\alert{提示:}\it\b 必须用定义计算$f'(x)$,
	$$\lim\limits_{\Delta x\to 0}\df{f(x+\Delta x)-f(x)}{\Delta x}
	=f(x)+ae^x.$$
	\pause 类似题目:\alert{\bf 习题7.2-7,辅导书(下)-P256-例5}
\end{frame}

\begin{frame}{解答题}
	\linespread{1.2}
	\ba{4.}\;某同学将乘积的导数公式错误地记作$(fg)'=f'g'$,然而在一次求导时
	居然得到了正确的结果。目前知道他使用的$f(x)=e^{x^2},(x>1/2)$,
	问他用到的$g(x)$可能是什么?
	
	\pause\alert{提示:}\it\b  
	$$\left(e^{x^2}g\right)'=\left(e^{x^2}\right)'g'$$
	从而$(2x-1)g'=2xg$,解得
	$$g=Ce^x\sqrt{2x-1}$$
\end{frame}

\begin{frame}{解答题}
	\linespread{1.5}
	\ba{5.}\;设二阶常系数线性微分方程
	$$y''+\alpha y'+\beta y=\gamma e^x$$
	的一个解为$y=e^{2x}+(1+x)e^x$,试确定其中的常数$\alpha,\beta,\gamma$.
	
	\pause\alert{提示:}\it\b $\alpha=-3,\beta=2,\gamma=-1$.
\end{frame}

\begin{frame}{解答题}
	\linespread{1.2}
	\ba{6.}\;求方程$y''+2y'+2y=2e^{-x}\cos^2\df x2$的通解.
	
	\pause\alert{提示:}\it\b 
	$$2e^{-x}\cos^2\df x2=e^{-x}+e^{-x}\cos x,$$
	利用叠加原理分别求解两个常系数非齐次线性微分方程。
\end{frame}

% \begin{frame}{解答题}
% 	\linespread{1.2}
% 	\alert{(2003考研)} 设$y(x)$在$\mathbb{R}$上具有二阶连续导数,
% 	$y'\ne 0$,$x=x(y)$为其反函数。
% 	\begin{enumerate}
% 	  \item 试将$x=x(y)$所满足的微分方程
% 	  $$\df{\d^2x}{\d y^2}+(y+\sin x)\left(\df{\d x}{\d y}\right)^3=0$$
% 	  变换为$y=y(x)$所满足的微分方程;
% 	  \item 求变换后的微分方程满足初始条件$y(0)=0$和$y'(0)=1.5$的解。
% 	\end{enumerate}
% 	
% 	\pause\alert{提示:}\it\b 
% 	$y''-y=\sin x\quad\Rightarrow\quad y=e^x-e^{-x}-\df12\sin x $
% \end{frame}

\begin{frame}{应用题}
	\linespread{1.2}
	\ba{7.}\;令$t=\tan x$,将方程
	$$\cos^4xy''_{xx}+2\cos^2x(1-\sin x\cos x)y'_x+y=e^{-\tan x}$$
	变换为$y$关于$t$的微分方程,并求其通解。
	
	\pause\alert{提示:}\it\b 
	$$y''_{tt}+2y'_t+y=e^{-t}$$
	$$y=\left(C_1+C_2\tan x+\df12\tan^2x\right)e^{-\tan x}$$
\end{frame}

\section{微分方程的应用}

\begin{frame}{应用题}
	\linespread{1.4}
	\ba{8.}\;已知某凹曲线任一点处的曲率为$\df1{2y^2\cos\alpha}$,其中
	$\alpha$为该点处的切线倾角($\cos\alpha>0$),且曲线在
	点$(1,1)$处的切线是水平的,求该曲线的方程。
	
	\pause\alert{提示:}\it\b $\cos\alpha>0=\df1{\sqrt{1+(y')^2}},
	\;y''>0$
	$$2y^2y''=[1+(y')^2]^2\quad\Rightarrow\quad
	4y=(x-1)^2+4$$
\end{frame}

% \begin{frame}{应用题}
% 	\linespread{1.2}
% 	\ba{(习题7.4-14)}一根挂在钉子上的链条,最初两端距离钉子
% 	分别为$8$m和$12$m,如不计钉子
% 	对链条产生的摩擦力,求链条从钉子上完全滑落所需的时间。
% 	
% 	\pause\alert{提示:}\it\b 设$x$为较长一端端点据钉子的距离
% 	$$\left\{\begin{array}{l}
% 		x''-\df g{10}x=-g\\
% 		x(0)=12\\
% 		x'(0)=0
% 	\end{array}\right.$$
% % 	\pause
% % 	\alert{思考:}若摩擦力等于$1$m长的链条的重量,模型又是怎样的?
% \end{frame}

\begin{frame}{应用题}
	\linespread{1.2}
	\ba{9.}\;已知某车间容积$V$,其空气中CO$_2$的密度为$\rho_1$,现以CO$_2$浓度$\rho_2(<<\rho_1)$的
	新鲜空气输入,问每分钟应输入多少才能在$T$分钟后使车间中CO$_2$的含量不超过$\rho_0$。
	(注:假设新注入的空气能够与原有空气立即混合达到均匀,且空气不会被压缩。)

	\pause\alert{提示:}\it\b 设$t$分钟时的$CO_2$含量为$C(t)$,
	$$C'=\rho_2V_1-C\df{V_1}{V},C(0)=\rho_1V$$
\end{frame}

\begin{frame}{应用题}
	\linespread{1.2}
	\ba{10.}\;某湖泊的水量为$V$,每年排入湖内的污水和净水量均为$V/6$,且湖内的总水量不变。
	已知1999年底湖内的污染物含量为$5m_0$。为了治理污染,从2000初开始,限定排入
	湖中的污水中污染物浓度不得超过$m_0/V$。问至少需要经过多少年,
	湖内的污染物含量能够降至$m_0$以下?(注:假设湖水中的污染物浓度时均匀分布的。)

	\pause\alert{提示:}\it\b 
	$$m'=\df{m_0}6-\df m3,\quad m=\df{m_0}2(1+9e^{-t/3})$$
	$t=6\ln3$年后,达到要求。
\end{frame} %常微分方程习题课

\end{document}
