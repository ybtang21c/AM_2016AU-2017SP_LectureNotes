\begin{center}
	{\Large\bf 选择题测试}
	
	{\it (60分钟,每题3分,总分:99+1)}
\end{center}

\begin{enumerate}
  \item 设$x,e^x,e^{2x}$分别为方程$y''+p(x)y'+q(x)y=f(x)$的三个解,则
  该方程满足$y(0)=1,y'(0)=3$的特解为
  (\underline{\hspace{1cm}})
  %\ps{A}
  
  (A)$2e^{2x}-e^x$\hspace{1cm}(B)$e^x-2e^{2x}$ \hspace{1cm}
  (C)$e^{2x}-e^x$\hspace{1cm}(D)$2e^{2x}-\cos x$
%   \begin{enumerate}[(A)]
%     \item $2e^{2x}-e^x$
%     \item $e^x-2e^{2x}$
%     \item $e^{2x}-e^x$
%     \item $2e^{2x}-\cos x$
%   \end{enumerate}
  \item 方程$y''+y=x^2+1+\sin x$的特解可设为
  (\underline{\hspace{1cm}})
  %\ps{C}
  \begin{enumerate}[(A)]
    \item $y^*=ax^2+bx+c+x(A\sin x+B\cos x)$
    \item $y^*=x(ax^2+bx+c+A\sin x+B\cos x)$
    \item $y^*=ax^2+bx+c+A\sin x$
    \item $y^*=ax^2+bx+c+A\cos x$
  \end{enumerate}
  \item 二重极限$\lim\limits_{(x,y)\to(0,0)}\df{xy^2}{x^2+y^4}=$
  (\underline{\hspace{1cm}})
  %\ps{D}  
  
  (A) $0$\hspace{1cm}(B) $1$  \hspace{1cm}(C)$\df12$\hspace{1cm}(D)不存在
  \item 函数$f(x,y)=x^2-ay^2$($a$为常数)在$(0,0)$处(\underline{\hspace{1cm}})
  %\ps{D}
  
  (A) 不取极值\quad(B)取极小值\quad (C)取极大值\quad
  (D)是否取极值取决于$a$的值
%   \begin{enumerate}[(A)]
%     \item 不取极值
%     \item 取极小值
%     \item 取极大值
%     \item 是否取极值取决于$a$的值
%   \end{enumerate}
  \item 函数$f(x,y)=\left\{\begin{array}{ll}
  0&,(x,y)=(0,0)\\
  \df{xy}{x^2+y^2}&,else
  \end{array}\right.$在原点处(\underline{\hspace{1cm}})
  %\ps{B}
  \begin{enumerate}[(A)]
    \item 连续且存在偏导数
    \item 不连续但存在偏导数
    \item 连续但不存在偏导数
    \item 不连续也不存在偏导数
  \end{enumerate}
  \item 函数$z=\sqrt{x^2+y^2}$在原点处 
  (\underline{\hspace{1cm}})
  %\ps{C}
  \begin{enumerate}[(A)]
    \item 偏导数和各方向的方向导数均存在
    \item 偏导数不存在,但各方向的方向导数均存在
    \item 偏导数和各方向的方向导数均不存在
    \item 偏导数存在,但某些方向的方向导数不存在
  \end{enumerate}
  \item 曲面$z=x+f(x-z)$的所有切平面都与某定直线
  (\underline{\hspace{1cm}})
  %\ps{B}
  
  (A) 垂直\hspace{1cm}(B) 平行  \hspace{1cm}(C)夹角为$\pi/4$\hspace{1cm}
  (D)夹角为$\pi/3$
%   \begin{enumerate}[(A)]
%     \item 垂直
%     \item 平行
%     \item 夹角为$\pi/4$
%     \item 夹角为$\pi/3$
%   \end{enumerate}
  \item 设$f(x,y)=(y-x^2)(y-x^4)$,$P(0,0),M(1,1)$,则(\underline{\hspace{1cm}})
%   \ps{B}
  \begin{enumerate}[(A)]
    \item $P,M$均为$f(x,y)$的极值点
    \item $P,M$均不是$f(x,y)$的极值点
    \item $P$为$f(x,y)$的极值点,$M$不是
    \item $M$为$f(x,y)$的极值点,$P$不是
  \end{enumerate}
  \item $x^2+y^2=1$时,$f(x,y)=(x^2+y^2)e^{-(x^2+y^2)}$ 
  (\underline{\hspace{1cm}})
%   \ps{B}
  
  (A)不取极值\hspace{1cm}(B)取极大值 \hspace{1cm}
  (C)取极小值\hspace{1cm}(D)取最大值
%   \begin{enumerate}[(A)]
%     \item 不取极值
%     \item 取极大值
%     \item 取极小值
%     \item 取最大值
%   \end{enumerate}
  \item $f(x,y)$在原点附近连续,$\lim\limits_{(x,y)\to(0,0)}
  \df{f(x,y)-|xy|}{(x^2+y^2)^2}=1$,则$f(x,y)$在原点
  (\underline{\hspace{1cm}})
%   \ps{C}
  
  (A)不取极值\hspace{1cm}(B)取极大值 \hspace{1cm}(C)取极小值\hspace{1cm}
  (D)不一定取极值
%   \begin{enumerate}[(A)]
%     \item 不取极值
%     \item 取极大值
%     \item 取极小值
%     \item 不一定取极值
%   \end{enumerate}
  \item $D$是顶点为$(1,0),(1,1),(2,0)$的三角区域,
  $I_k=\ds\iint_D[\ln(x+y)]^k\d\sigma$,则(\underline{\hspace{1cm}})
%   \ps{B}
  
  (A)$I_1<I_2<I_3$\quad(B)$I_1>I_2>I_3$\quad
  (C)$I_1<I_3<I_2$\quad(D)$I_3<I_1<I_2$
%   \begin{enumerate}[(A)]
%     \item $I_1<I_2<I_3$
%     \item $I_1>I_2>I_3$
%     \item $I_1<I_3<I_2$
%     \item $I_3<I_1<I_2$
%   \end{enumerate}
  \item 设$D_1:x+y\leq1,x\geq 0,y\geq0;\,D_2:|x|+|y|\leq1$,
  $I_j=\ds\iint_{D_j}e^{|x|+|y|}\d\sigma(j=1,2)$,则
  (\underline{\hspace{1cm}})
%   \ps{C}
  
  (A)$I_1=I_2$\hspace{1cm}(B)$2I_1=I_2$\hspace{1cm}
  (C)$4I_1=I_2$\hspace{1cm}(D)$I_1=4I_2$
%   \begin{enumerate}[(A)]
%     \item $I_1=I_2$
%     \item $2I_1=I_2$
%     \item $4I_1=I_2$
%     \item $I_1=4I_2$
%   \end{enumerate}
  \item 设$\ds\iint_{x^2+y^2\leq a^2}\sqrt{a^2-x^2-y^2}\d\sigma=\pi$,
  则$a=$(\underline{\hspace{1cm}})
%   \ps{D}
  
  (A)$1$\hspace{1cm}(B)$\sqrt[3]{\df12}$\hspace{1cm}
  (C)$\sqrt[3]{\df34}$\hspace{1cm}(D)$\sqrt[3]{\df32}$
%   \begin{enumerate}[(A)]
%     \item $1$
%     \item $\sqrt[3]{\df12}$
%     \item $\sqrt[3]{\df34}$
%     \item $\sqrt[3]{\df32}$
%   \end{enumerate}
  \item 设$\Omega$为$z\geq{\sqrt{x^2+y^2}}(z\geq0)$介于$z=1$和$z=2$之间
  的部分,则$\ds\iiint_{\Omega}f(x^2+y^2+z^2)\d V=$
  (\underline{\hspace{1cm}})
%   \ps{A}
  \begin{enumerate}[(A)]
    \item $\dint_1^2\dint_0^{2\pi}\dint_0^zf(r^2+z^2)r\d r\d\theta\d z$
    \item $\dint_0^{2\pi}\dint_1^2r\dint_0^1f(r^2+z^2)\d z\d r\d\theta$
    \item $\dint_0^{2\pi}\dint_0^{\frac{\pi}4}\dint_1^2
    f(r^2)r^2\sin\varphi\d r\d\varphi\d\theta$
    \item $\dint_0^{2\pi}\dint_{\frac{\pi}2}^{\frac{\pi}4}\dint_1^2
    f(r^2)r^2\sin\varphi\d r\d\varphi\d\theta$
  \end{enumerate}
  \item 设$\Gamma$为上半圆周$x^2+y^2=2x$从原点到$(1,1)$的部分,则
  $\dint_{\Gamma}P(x,y)\d x+Q(x,y)\d y=$(\underline{\hspace{1cm}})
%   \ps{C}
  \begin{enumerate}[(A)]
    \item $\dint_{\Gamma}\left[P(x,y)(x-1)+Q(x,y)\sqrt{2x-x^2}\right]\d s$
    \item $\dint_{\Gamma}\left[P(x,y)(1-x)-Q(x,y)\sqrt{2x-x^2}\right]\d s$
    \item $\dint_{\Gamma}\left[P(x,y)\sqrt{2x-x^2}+Q(x,y)(1-x)\right]\d s$
    \item $\dint_{\Gamma}\left[-P(x,y)\sqrt{2x-x^2}+Q(x,y)(x-1)\right]\d s$
  \end{enumerate}
  \item 设$f(x)$可微,$f(0)=1$,则$\lim\limits_{t\to0^+}\df1{\pi t^3}
  \ds\iint_{x^2+y^2\leq t^2}  f(\sqrt{x^2+y^2})\d x\d y=$ 
  (\underline{\hspace{1cm}})
%   \ps{C}
  
  (A)$0$\hspace{1cm}(B)$\df23f'(0)$\hspace{1cm}
  (C)$+\infty$\hspace{1cm}(D)不存在但也不是$\infty$
%   \begin{enumerate}[(A)]
%     \item $0$
%     \item $\df23f'(0)$
%     \item $+\infty$
%     \item 不存在但也不是$\infty$
%   \end{enumerate}
  \item 设$\Gamma$为$A(-1,0),B(-3,2)$和$C(3,0)$为顶点的三角形区域沿$ABCA$方向
  的封闭折线,则$\ds\oint_{\Gamma}(3x-y)\d x+(x-2y)\d y=$
  (\underline{\hspace{1cm}})
%   \ps{D}
  
  (A)$16$\hspace{1cm}(B)$-16$\hspace{1cm}
  (C)$8$\hspace{1cm}(D)$-8$
%   \begin{enumerate}[(A)]
%     \item $16$
%     \item $-16$
%     \item $8$
%     \item $-8$
%   \end{enumerate}
  \item 设$S$为单位球的外侧,$S_1$为其上半部分,则下列等式成立的是
  (\underline{\hspace{1cm}})
%   \ps{A}
  \begin{enumerate}[(A)]
    \item $\ds\iint_{S}|z|\d S=2\ds\iint_{S_1}|z|\d S$
    \item $\ds\iint_{S}|z|\d x\d y=2\ds\iint_{S_1}|z|\d x\d y$
    \item $\ds\iint_{S}|y|\d x\d y=2\ds\iint_{S_1}|y|\d x\d y$
    \item $\ds\iint_{S}|x|\d x\d y=2\ds\iint_{S_1}|x|\d x\d y$
  \end{enumerate}
  \item 设$S$为$z=\sqrt{x^2+y^2}$被$z=1$所截得的有限部分的外侧,则
  $\ds\iint_S x\d y\d z+y\d z\d x+(z^2-2z)\d x\d y=$
  (\underline{\hspace{1cm}})
%   \ps{D}
  
  (A)$-\df32\pi$\hspace{1cm}(B)$0$\hspace{1cm}
  (C)$\df{\pi}2$\hspace{1cm}(D)$\df32\pi$
%   \begin{enumerate}[(A)]
%     \item $-\df32\pi$
%     \item $0$
%     \item $\df{\pi}2$
%     \item $\df32\pi$
%   \end{enumerate}
  \item 设$L_1:\df{x^2}{4}+\df{y^2}{9}=1,L_2:\df{x^2}{9}+\df{y^2}{4}=1$,
  二者所围封闭区域分别为$D_1,D_2$,则下列正确的是
  (\underline{\hspace{1cm}})
%   \ps{C}
  \begin{enumerate}[(A)]
    \item $\dint_{L_1}(x+y^2)\d s=2\dint_{L_2}y^2\d s$
    \item $\dint_{L_1}(x^2+y)\d s=2\dint_{L_2}(x^2+y)\d s$
    \item $\ds\iint_{D_1}(x+y^3)\d\sigma=2\ds\iint_{D_2}(x+y^3)\d\sigma$
    \item $\ds\iint_{D_1}(x^2+y)\d\sigma=2\ds\iint_{D_2}(x^2+y)\d\sigma$
  \end{enumerate}
  \item $f(x,y)$偏导连续,曲线$L:f(x,y)=1$过第二象限的点$M$
  和第四象限的点$N$,$\Gamma$为$L$上从$M$到$N$的一段弧,则下列
  小于零的是
  (\underline{\hspace{1cm}})
%   \ps{B}
  \begin{enumerate}[(A)]
    \item $\dint_{\Gamma}f(x,y)\d x$
    \item $\dint_{\Gamma}f(x,y)\d y$
    \item $\ds\int_{\Gamma}f(x,y)\d s$
    \item $\ds\int_{\Gamma}f\,'_x(x,y)\d x+f\,'_y(x,y)\d y$
  \end{enumerate}
  \item 设曲面$S_1:x^2+y^2+z^2=1(z\geq
  0)$,$S_2$为$S_1$在第一卦限中的部分,
  则以下正确的是
  (\underline{\hspace{1cm}})
%   \ps{C}
  \begin{enumerate}[(A)]
    \item $\ds\iint_{S_1}x\d S=4\iint_{S_2}x\d S$
    \item $\ds\iint_{S_1}y\d S=4\iint_{S_2}x\d S$
    \item $\ds\iint_{S_1}z\d S=4\iint_{S_2}x\d S$
    \item $\ds\iint_{S_1}xyz\d S=4\iint_{S_2}xyz\d S$
  \end{enumerate}
  \item 设$f(r)$二阶连续可微,$r=\sqrt{x^2+y^2+z^2}$,
  若$\mathrm{div}(\bigtriangledown\,f(r))=0$,则$f(r)=$
  (\underline{\hspace{1cm}})
%   \ps{B}
  
  (A)$C_1r+C_2$\hspace{1cm}(B)$C_1/r+C_2$\hspace{1cm}
  (C)$C_1r^2+C_2$\hspace{1cm}(D)$C_1/r^2+C_2$
%   \begin{enumerate}[(A)]
%     \item $C_1r+C_2$
%     \item $C_1/r+C_2$
%     \item $C_1r^2+C_2$
%     \item $C_1/r^2+C_2$
%   \end{enumerate}
  以上$C_1,C_2$为任意常数
  \item 设$\Gamma$是从原点沿折线$y=1-|x-1|$至点$(2,0)$,则
  $\dint_{\Gamma}-y\d x+x\d y=$
  (\underline{\hspace{1cm}})
%   \ps{D}
  
  (A)$0$\hspace{1cm}(B)$-1$\hspace{1cm}
  (C)$2$\hspace{1cm}(D)$-2$
%   \begin{enumerate}[(A)]
%     \item 0
%     \item -1
%     \item 2
%     \item -2
%   \end{enumerate}
  \item 设$S$是三个坐标面与平面$x=a,y=b,z=c$(其中$a,b,c$均大于零)所围成的
  封闭曲面的外侧,则$\ds\oiint_S(x^2-yz)\d y\d z+(y^2-zx)\d z\d x
  +(z^2-xy)\d x\d y=$
  (\underline{\hspace{1cm}})
%   \ps{A}
  \begin{enumerate}[(A)]
    \item $abc(a+b+c)$
    \item $a^2b^2c^2(a+b+c)$
    \item $ab+ac+bc$
    \item $(a+b+c)^2$
  \end{enumerate}
  \item 若$(x^4+4xy^3)\d x+(ax^2y^2-5y^4)\d y$为全微分,则其原函数为
  (\underline{\hspace{1cm}})
%   \ps{C}
  \begin{enumerate}[(A)]
    \item $\df15x^5+3x^2y^2-y^5+C$
    \item $\df15x^5+4x^2y^2-5y^4+C$
    \item $\df15x^5+2x^2y^3-y^5+C$
    \item $\df15x^5+2x^2y^3-5y^4+C$
  \end{enumerate}
  \item 设$\sumn(-1)^n\df{(x-a)^n}n$在$x>0$时发散,在$x=0$处收敛,则$a=$
  (\underline{\hspace{1cm}})
%   \ps{B}
  
  (A)$1$\hspace{1cm}(B)$-1$\hspace{1cm}
  (C)$2$\hspace{1cm}(D)$-2$
%   \begin{enumerate}[(A)]
%     \item 1
%     \item -1
%     \item 2
%     \item -2
%   \end{enumerate}
  \item 设$\sumn a_n(x-1)^n$在$x=-1$处收敛,则它在$x=2$处
  (\underline{\hspace{1cm}})
%   \ps{B}
  
  (A)发散\hspace{1cm}(B)绝对收敛\hspace{1cm}
  (C)条件收敛\hspace{1cm}(D)敛散性与$a_n$有关
%   \begin{enumerate}[(A)]
%     \item 发散
%     \item 绝对收敛
%     \item 条件收敛
%     \item 敛散性与$a_n$有关
%   \end{enumerate}
  \item 设$\sumn a_nx^n$和$\sumn b_nx^n$收敛半径均为$R$,
  $\sumn(a_n+b_n)x^n$收敛半径为$R_1$,则
  (\underline{\hspace{1cm}})
%   \ps{C}
  
  (A)$R=R_1$\hspace{1cm}(B)$R>R_1$\hspace{1cm}
  (C)$R\leq R_1$\hspace{1cm}(D)$R\geq R_1$
  \item 若级数$\sumn a_n$条件收敛,则$\sumn na_n(x-1)^n$在$x=\sqrt3$和$x=3$分别
  (\underline{\hspace{1cm}})
%   \ps{B}
  
  (A)收敛,收敛\hspace{1cm}(B)收敛,发散\hspace{1cm}
  (C)发散,收敛\hspace{1cm}(D)发散,发散
%   \begin{enumerate}[(A)]
%     \item $R=R_1$
%     \item $R>R_1$
%     \item $R\leq R_1$
%     \item $R\geq R_1$
%   \end{enumerate}
  \item 设$f(x)=\left\{\begin{array}{ll}
  x&,x\in[0,1/2]\\ 2(1-x)&,x\in(1/2,1)
  \end{array}\right.$,$S(x)=\df{a_0}2+\sumn a_n\cos n\pi x,
  (x\in\mathbb{R})$,其中$a_n=2\dint_0^1f(x)\cos n\pi x\d x
  (n=0,1,2,\ldots)$,则$S\left(-\df52\right)=$
  (\underline{\hspace{1cm}})
%   \ps{A}
  
  (A)$\df34$\hspace{1cm}(B)$\df12$\hspace{1cm}
  (C)$-\df34$\hspace{1cm}(D)$-\df12$
%   \begin{enumerate}[(A)]
%     \item $\df34$
%     \item $\df12$
%     \item $-\df34$
%     \item $-\df12$
%   \end{enumerate}
  \item 将函数$f(x)=\left\{\begin{array}{ll}
  1&,0\leq x<1\\ x+1&,1\leq x\leq \pi
  \end{array}\right.$在$[0,\pi]$上展开成余弦级数,则其和函数在$x=1$和$x=\pi$
  处的值分别为
  (\underline{\hspace{1cm}})
%   \ps{A}
  
  (A)$\df32,\pi+1$\hspace{1cm}(B) $2,0$\hspace{1cm}
  (C)$2,\pi+1$\hspace{1cm}(D)$\df32,\df{\pi}2+1$
%   \begin{enumerate}[(A)]
%     \item $\df32,\pi+1$
%     \item $2,0$
%     \item $2,\pi+1$
%     \item $\df32,\df{\pi}2+1$
%   \end{enumerate}
  \item 设$f(x)$连续,则$\dint_0^1\d y\dint_{-\sqrt{1-y^2}}^{1-y}f(x,y)\d y=$
  (\underline{\hspace{1cm}})
%   \ps{D}
  \begin{enumerate}[(A)]
    \item $\dint_0^1\d x\dint_{0}^{x-1}f(x,y)\d y
    +\dint_{-1}^0\d x\dint_0^{\sqrt{1-x^2}}f(x,y)\d y$
    \item $\dint_0^1\d x\dint_{0}^{1-x}f(x,y)\d y
    +\dint_{-1}^0\d x\dint_{-\sqrt{1-x^2}^0}f(x,y)\d y$
    \item $\dint_0^{\frac{\pi}2}\d\theta\dint_0^{\frac1{\cos\theta+\sin\theta}}
    f(r\cos\theta,r\sin\theta)\d r+\dint_{\frac{\pi}2}^{\pi}\d\theta
    \dint_0^{\frac1{\cos\theta+\sin\theta}}f(r\cos\theta,r\sin\theta)\d r$
    \item $\dint_0^{\frac{\pi}2}\d\theta\dint_0^{\frac1{\cos\theta+\sin\theta}}
    f(r\cos\theta,r\sin\theta)r\d r+\dint_{\frac{\pi}2}^{\pi}\d\theta
    \dint_0^{\frac1{\cos\theta+\sin\theta}}f(r\cos\theta,r\sin\theta)r\d r$
  \end{enumerate}
  
\end{enumerate}

ACDDB CBBBC BCDAC CDADC BCBDA CBBCB AAD