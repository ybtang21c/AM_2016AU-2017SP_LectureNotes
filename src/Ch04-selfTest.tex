\begin{center}
	{\Large\bf 导数、微分的概念与性质自测}
	
	(时间:60分钟)
\end{center}

\bigskip

{\bf 一、选择题}(每题3分)

\bigskip

1、设$f(x)=\left\{\begin{array}{ll}
\df{1-\cos x}{\sqrt x},& x>0\\ x^2g(x),& x\leq 0
\end{array}\right.$,其中$g(x)$有界,则$f(x)$在$x=0$处(\quad)%D
\begin{tabbing}
	\hspace{8cm}\=\kill
	\quad\quad\quad(A)\;极限不存在 \> 
	(B)\;极限存在,但不连续 \\ 
	\quad\quad\quad(C)\;连续,但不可导\>
	(D)\;可导
\end{tabbing}

\bigskip

2、函数$f(x)$可导,且曲线$y=f(x)$在点$(x_0,f(x_0))$处的切线与直线$y=2-x$
垂直,则当$\Delta x\to0$时,该函数在$x=x_0$处的微分$\d y$是(\quad)%B
\begin{tabbing}
	\hspace{8cm}\=\kill
	\quad\quad\quad(A)\;与$\Delta x$同阶但不等价的无穷小 \> 
	(B)\;与$\Delta x$等价的无穷小 \\ 
	\quad\quad\quad(C)\;$\Delta x$的高阶无穷小\>
	(D)\;比$\Delta x$低阶的无穷小
\end{tabbing}

\bigskip

3、函数$f(x)=\ln|x-1|$的导数是(\quad)%B
\begin{tabbing}
	\hspace{8cm}\=\kill
	\quad\quad\quad(A)\;$\df1{|x-1|}$ \> 
	(B)\;$\df1{x-1}$ \\ 
	\quad\quad\quad(C)\;$\df1{1-x}$\>
	(D)\;$\left\{\begin{array}{ll}
	\df1{x-1},&x>1\\ \df1{1-x},&x<1
	\end{array}\right.$
\end{tabbing}

\bigskip

4、函数$y=\df13x^3+\df12x^2+6x+1$的图形在点$(0,1)$处的切线与$x$轴的交点
坐标是(\quad)%B

\quad (A)\;$(-1,0)$\quad\quad\quad(B)\;$(-1/6,0)$
\quad\quad\quad (C)\;$(1,0)$\quad\quad\quad(D)\;$(1/6,0)$

\bigskip

5、函数$f(x)=\left|\df{\sin x}x\right|$在$x=\pi$处(\quad)%D
\begin{tabbing}
	\hspace{8cm}\=\kill
	\quad\quad\quad(A)\;$f'_+(\pi)=-1/\pi$ \> 
	(B)\;$f'(\pi)=1/\pi$ \\ 
	\quad\quad\quad(C)\;$f'_-(\pi)=1/\pi$\>
	(D)\;$f'_+(\pi)=1/\pi$
\end{tabbing}

\bigskip

6、$f(x)=\lim\limits_{n\to\infty}\sqrt[n]{1+|x|^{3n}}$在
$(-\infty,+\infty)$内(\quad)%B
\begin{tabbing}
	\hspace{8cm}\=\kill
	\quad\quad\quad(A)\;处处可导 \> 
	(B)\;恰有一个不可导点 \\ 
	\quad\quad\quad(C)\;恰有两个不可导点\>
	(D)\;至少有三个不可导点
\end{tabbing}

\bigskip

7、设$f(x)$在$x=0$连续,其$\limx0\df{f(x^2)}{x^2}=1$,则(\quad)%C
\begin{tabbing}
	\hspace{8cm}\=\kill
	\quad\quad\quad(A)\;$f(0)=0$,且$f'(0)$存在 \> 
	(B)\;$f(0)=1$,且$f'_+(0)$存在 \\ 
	\quad\quad\quad(C)\;$f(0)=0$,且$f'_+(0)$存在\>
	(D)\;$f(0)=1$,且$f'(0)$存在
\end{tabbing}

\bigskip

8、设函数$f(x)$以$4$为周期,且处处可导,$\limx0\df{f(1)-f(1-x)}{2x}=1$,则
曲线$y=f(x)$在点$(5,f(5))$处的法线斜率为(\quad)%A

\quad (A)\;$1/2$\quad\quad\quad(B)\;$0)$
\quad\quad\quad (C)\;$-2$\quad\quad\quad(D)\;$-1$

\bigskip

9、若函数$f(x)$在$x_0$可导,则$|f(x)|$在$x_0$处(\quad)%B
\begin{tabbing}
	\hspace{8cm}\=\kill
	\quad\quad\quad(A)\;可导 \> 
	(B)\;连续,但未必可导 \\ 
	\quad\quad\quad(C)\;连续,但不可导\>
	(D)\;未必连续
\end{tabbing}

\bigskip

10、函数$f(x)=\left\{\begin{array}{ll}
\sqrt{|x|}\sin\df1{x^2},& x\ne 0\\
0,& x=0
\end{array}\right.$在$x=0$处(\quad)%C
\begin{tabbing}
	\hspace{8cm}\=\kill
	\quad\quad\quad(A)\;极限不存在 \> 
	(B)\;极限存在,但不连续 \\ 
	\quad\quad\quad(C)\;连续,但不可导\>
	(D)\;可导
\end{tabbing}

\bigskip

11、设$F(x)=\left\{\begin{array}{ll}
\df{f(x)}x,& x\ne0\\ f(0),& x=0
\end{array}\right.$,其中$f(0)=0$,$f'(0)\ne 0$,则
$x=0$是$F(x)$的(\quad)%B
\begin{tabbing}
	\hspace{8cm}\=\kill
	\quad\quad\quad(A)\;连续点 \> 
	(B)\;第一类间断点 \\ 
	\quad\quad\quad(C)\;第二类间断点\>
	(D)\;连续点还是间断点无法确定
\end{tabbing}

\bigskip

12、设$f(x)=\left\{\begin{array}{ll}
x^{\frac53}\sin\df1x,& x\ne0\\ 0,& x=0
\end{array}\right.$,$f(x)$在$x=0$处(\quad)%C
\begin{tabbing}
	\hspace{8cm}\=\kill
	\quad\quad\quad(A)\;不连续 \> 
	(B)\;连续,但不可导 \\ 
	\quad\quad\quad(C)\;可导,但导函数不连续\>
	(D)\;导函数连续
\end{tabbing}

\bigskip

13、$f(x)$在原点附近有定义,且满足$|f(x)|\leq x^2$,则$x=0$是
$f(x)$的(\quad)%C
\begin{tabbing}
	\hspace{8cm}\=\kill
	\quad\quad\quad(A)\;间断点 \> 
	(B)\;连续但不可导点 \\ 
	\quad\quad\quad(C)\;可导点且$f'(0)=0$\>
	(D)\;可导点但$f'(0)$的值不确定
\end{tabbing}

\bigskip

14、设$f(x)$可导,$F(x)=f(x)(1+|\sin x|)$,若使$F(x)$在$x=0$
可导,则(\quad)%A
\begin{tabbing}
	\hspace{8cm}\=\kill
	\quad\quad\quad(A)\;$f(0)=0$ \> 
	(B)\;$f'(0)=0$ \\ 
	\quad\quad\quad(C)\;$f(0)+f'(0)=0$\>
	(D)\;$f(0)-f'(0)=0$
\end{tabbing}

% 15、设$f(x)=\left\{\begin{array}{ll}
% \df{g(x)}x,& x\ne0\\ f(0),& x=0
% \end{array}\right.$,其中
% $g(x)$在$x=0$处二阶可导,且$g(0)=g'(0)=0$,则$f(x)$在$x=0$处(\quad)%D
% \begin{tabbing}
% 	\hspace{8cm}\=\kill
% 	\quad\quad\quad(A)\;不连续 \> 
% 	(B)\;连续,但不可导 \\ 
% 	\quad\quad\quad(C)\;可导,但导函数不连续\>
% 	(D)\;导函数连续
% \end{tabbing}

\bigskip

15、当$x>0$时,曲线$y=x\sin\df1x$(\quad)%A
\begin{tabbing}
	\hspace{8cm}\=\kill
	\quad\quad\quad(A)\;仅有水平渐近线 \> 
	(B)\;仅有铅直渐近线 \\ 
	\quad\quad\quad(C)\;同时有水平和铅直渐近线\>
	(D)\;无渐近线
\end{tabbing}

\bigskip

16、曲线$y=e^{\frac1{x^2}}\arctan\df{x^2+x+1}{(x-1)(x+2)}$的渐近线
有
(\quad)%B

\quad (A)\;$1$条\quad\quad\quad(B)\;$2$条
\quad\quad\quad (C)\;$3$条\quad\quad\quad(D)\;$4$条

\bigskip

17、设$f(x)=(e^x-1)(e^{2x}-2)\ldots(e^{nx}-n)$,其中$n$为正整数,则
$f'(0)=$(\quad)%A
\begin{tabbing}
	\hspace{8cm}\=\kill
	\quad\quad\quad(A)\;$(-1)^{n-1}(n-1)!$ \> 
	(B)\;$(-1)^{n}(n-1)!$ \\ 
	\quad\quad\quad(C)\;$(-1)^{n-1}n!$\>
	(D)\;$(-1)^{n}n!$
\end{tabbing}

\bigskip

18、$\left.\left(\sin\df x2+\cos2x\right)^{(99)}\right|_{x=\pi}=$
(\quad)%A

\quad (A)\;$0$\quad\quad\quad(B)\;$-\df1{2^{99}}$
\quad\quad\quad (C)\;$2^{99}-\df1{2^{99}}$\quad\quad\quad(D)\;$2^{99}$

\bigskip

19、设$f(x)=x^3+2x^2+4x,\;(x>0)$,$g(x)$为其反函数,则$g'(7)=$(\quad)%B

\quad (A)\;$10$\quad\quad\quad(B)\;$\df1{10}$
\quad\quad\quad (C)\;$179$\quad\quad\quad(D)\;$\df1{179}$

\bigskip

20、$f(x)=(1-e^x)|x^3-x|$的不可导点个数为(\quad)%B

\quad (A)\;$0$\quad\quad\quad(B)\;$1$
\quad\quad\quad (C)\;$2$\quad\quad\quad(D)\;$3$

\bigskip

{\bf 二、填空题}(每题4分)

\bigskip

% 1、设$f(x)=\left\{\begin{array}{ll}
% \df{\ln(1+bx)}x,&x\ne 0\\-1,& x=0
% \end{array}\right.$,已知$1+bx>0$,则当$f(x)$在$x=0$可导时
% $f'(0)=$\underline{\hspace{4cm}}.%$-\df12$

1、若$f(t)=\limx{\infty}t\left(1+\df1x\right)^{2tx}$,则$f'(t)=$
\underline{\hspace{4cm}}.%$te^{2t}$

% 2、$\left(e^{\tan\frac1x}\sin\df1x\right)'=$

\bigskip

2、曲线$\left\{\begin{array}{l}
x=e^t\sin 2t\\ y=e^t\cos t
\end{array}\right.$在点$(0,1)$处的法线方程为\underline{\hspace{4cm}}.
%$y+2x-1=0$

\bigskip

3、$f(x)$在$x=2$附近可导,且$f'(x)=e^{f(x)}$,$f(2)=1$,则$f^{(n)}(2)=$
\underline{\hspace{4cm}}.%$(n-1)!e^{nf(x)}$

\bigskip

4、$f(x)=x^n$在$(1,1)$处的切线与$x$轴交于$x_n$,则
$\limn f(x_n)=$\underline{\hspace{4cm}}.%$e^{-1}$

\bigskip

5、设$y=y(x)$是由$\sin xy=\ln\df{x+e}y+1$确定的隐函数,则$y''(0)=$
\underline{\hspace{4cm}}.%$e^3(3e^3-4)$

\bigskip

6、设$y=\ln(1+3^{-x})$,则$\d y=$\underline{\hspace{4cm}}.
%$-\df{\ln3}{1+3^x}\d x$

\bigskip

7、$f(x)$在$x=a$可导,且$f(a)\ne 0$,则
$\limn\left[\df{f\left(a+\df1n\right)}{f(a)}\right]^n=$
\underline{\hspace{4cm}}.%$e^{\frac{f'(a)}{f(a)}}$

\bigskip

8、$f(x)$在$x=1$连续可导,$f'(1)=2$,则
$\limx{0}\df1x\df{\d}{\d x}f(\cos^22x)=$
\underline{\hspace{4cm}}.%$-16$

\bigskip

9、$f(x)$是以$5$为周期的连续函数,且在原点附近满足
$$f(1+\sin x)-3f(1-\sin x)=8x+\circ(x),$$
已知$f(x)$在$x=1$可导,则$y=f(x)$在$x=6$处的切线斜率为
\underline{\hspace{4cm}}.%$2$

\bigskip

10、$\left(\df{x+b}{x+a}\right)^{(n)}=$
\underline{\hspace{4cm}}.%$\df{(-1)^nn!(b-a)^a}{(x+a)^{n+1}}$

\newpage

\begin{center}
	{\Large\bf 参考解答}
\end{center}

{\bf 一、选择题}

DBBBD\quad CCABC\quad BCCAA\quad BAABC

{\bf 二、填空题}

1、$te^{2t}(1+2t)$

2、$y+2x-1=0$

3、$(n-1)!e^{n}$

4、$e^{-1}$

5、$-e^4+e^3-e^2+2e$

6、$-\df{\ln3}{1+3^x}\d x$

7、$e^{\frac{f'(a)}{f(a)}}$

8、$-8$

9、$2$

10、$\df{(-1)^nn!(b-a)}{(x+a)^{n+1}}$