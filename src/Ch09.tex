\setcounter{chapter}{8}

\chapter{向量值函数}

\section{向量值函数及其极限与连续}

$$f:\mathbb{R}\to\mathbb{R}^n$$

例如:
\begin{itemize}
  \setlength{\itemindent}{1cm}
  \item $\bm{r}(t)=(t\sin t, t\cos t),\;t\in\mathbb{R}^+$ 
  \item
  $\bm{r}(t)=(a+mt)\bm{i}+(b+nt)\bm{j}+(c+pt)\bm{k},\;t\in\mathbb{R}$ 
\end{itemize}

{\bf 注:}向量值函数就是曲线的参数方程。

\subsection{极限与连续}

已知向量值函数$\bm{r}(t)=(f(t),g(t))$在$t_0$附近有定义 
\begin{enumerate}[(1)]
  \setlength{\itemindent}{1cm}
  \item {\bf $\lim\limits_{t\to t_0}\bm{r}(t)=(a,b)$:}
  $$\lim\limits_{t\to t_0}f(t)=a,\;\lim\limits_{t\to t_0}g(t)=b$$
  \item {\bf $\bm{r}(t)$在$t_0$处连续:}
  $$\bm{r}(t_0)=\lim\limits_{t\to t_0}\bm{r}(t)$$
\end{enumerate}

{\bf P73-例3:}设$\bm{r}(t)=\df{\sin 2t}{t}\bm{i}+\ln(1+t)\bm{j}$,求
$\lim\limits_{t\to 0}\bm{r}(t)$。

{\bf  P73-例4:}设$\bm{r}(t)=e^{-t}\cos 2t\bm{i}+e^{-t}\sin 2t
\bm{j}+e^{-t}\bm{k}$,求$\lim\limits_{t\to +\infty}\bm{r}(t)$。

\section{向量值函数的导数与微分}

$$\bm{r}'(t_0)=\lim\limits_{\Delta t\to 0}
\df{\bm{r}(t_0+\Delta t)-\bm{r}(t_0)}{\Delta t}$$

{\bf 注:}
\begin{enumerate}[(1)]
  \setlength{\itemindent}{1cm}
  \item $\bm{r}'(t)$是一个向量;
  \item $\bm{r}(t)$在$t_0$可导,则其每个分量函数在$t_0$可导
  \item $\bm{r}(t)$在$t_0$可导,则$\bm{r}(t)$在$t_0$连续 
  \item $\bm{r}'(t_0)$的几何意义:对应曲线在$t_0$处的切向量
\end{enumerate}

{\bf 曲线$\bm{r}(t)$在区间$I$上光滑:}
\begin{enumerate}[(1)]
  \setlength{\itemindent}{1cm}
  \item $\bm{r}'(t)$在$I$上连续
  \item $\bm{r}'(t)\ne \bm{0}$
\end{enumerate}

{\bf P78-例2:}判断曲线$\bm{r}(t)=(1+t^3,t^2)$是否为光滑曲线?

{\bf 注:}二维向量值函数$\bm{r}(t)$可导,不等价于对应函数$y=f(x)$可导。

\begin{itemize}
  \setlength{\itemindent}{1cm}
  \item {\bf 向量值函数的微分:}
  $$\df{\d\bm{r}}{\d t}=\bm{r}'(t)\;\Leftrightarrow\;
  \d\bm{r}=\bm{r}'(t)\d t$$
  \item {\bf 向量值函数的高阶导数:}
  $$\bm{r}^{(n)}(t_0)=\left.\df{\d\bm{r}^{(n-1)}(t)}{\d t}\right|_{t=t_0}$$
\end{itemize}

{\bf 求导法则:}

\begin{enumerate} 
  \item $(\bm{C})'=\bm{0}$ 
  \item $[\bm{u}(t)\pm\bm{v}(t)]'=\bm{u}'(t)\pm\bm{v}'(t)$ 
  \item $[k\bm{u}(t)]'=k\bm{u}'(t)$ 
  \item $[f(t)\bm{u}(t)]'=f'(t)\bm{u}(t)+f(t)\bm{u}'(t)$ 
  \item {$[\bm{u}(t)\cdot\bm{v}(t)]'=\bm{u}'(t)\cdot\bm{v}(t)
  +\bm{u}(t)\cdot\bm{v}'(t)$} 
  \item {$[\bm{u}(t)\times\bm{v}(t)]'=\bm{u}'(t)\times\bm{v}(t)
  +\bm{u}(t)\times\bm{v}'(t)$} 
  \item $[\bm{u}(f(t))]'=\bm{u}'(f(t))f'(t)$ 
\end{enumerate}

{\bf P79-例4:}设$\bm{r}(t)$是可导的向量值函数,且$\bm{r}'(t)\ne 0$,若
$|\bm{r}(t)|=C$,证明:$\bm{r}(t)$与$\bm{r}'(t)$垂直。

{\bf P79-例5:}设质量为$m$的质点的位置为$\bm{r}(t)$,速度和加速度分别为
$\bm{v}(t)$和$\bm{a}(t)$,则其角动量$\bm{L}(t)
=m\bm{r}(t)\times\bm{v}(t)$,转动力矩$\bm{M}(t)=m\bm{r}(t)
\times\bm{a}(t)$,证明:$\bm{L}'(t)=\bm{M}(t)$。

{\bf 空间曲线的切线与法平面}

{\bf P80-例6:}设空间曲线的参数方程为$\Gamma:x=t,y=t^2,z=t^3$在点
$(1,1,1)$处的切线方程与法平面方程。

\section{向量值函数的不定积分与定积分}

设向量值函数$\bm{r}(t)$在区间$I$上有定义 
\begin{enumerate}
  \item {\bf $\bm{R}(t)$是$\bm{r}(t)$在$I$上的一个原函数:} 
  $$\bm{R}'(t)=\bm{r}(t)\;(t\in I)$$ 
  \item {\bf $\bm{r}(t)$的不定积分:}
  $$\dint \bm{r}(t)\d t=\bm{R}(t)+\bm{C}\;(\bm{C}\in\mathbb{R}^n)$$
\end{enumerate}

\begin{enumerate}
  \item {\bf 线性性} 
  \item {\bf 区间可加性} 
  \item $\dint\bm{r}(t)\d t=\left(\dint f(t)\d t\right)\bm{i}
  +\left(\dint g(t)\d t\right)\bm{j}+\left(\dint h(t)\d t\right)\bm{k}$ 
  \item $\dint_a^b\bm{r}(t)\d t=\bm{R}(t)|_a^b=\bm{R}(b)-\bm{R}(a)$
\end{enumerate}

{\bf P86-例6:}求解关于向量值函数$\bm{r}(t)$的微分方程
$$\df{\d\bm{r}}{\d t}=\df 32(t+1)^{1/2}\bm{i}+e^{-t}\bm{j},\;\bm{r}(0)=0$$

{\bf 例:}一枚导弹以初始速度$\bm{v}_0$,仰角$\alpha$发射,假设导弹只受重力作用,
空气阻力可以忽略不记,求导弹的位置函数$\bm{r}(t)$,并问$\alpha$
取何值时其射程最远。

\newpage

\section*{课后作业}

{\bf 【基本题】}

\begin{itemize}
  \setlength{\itemindent}{1cm}
  \item 习题9.1:6(2,4)
  \item 习题9.2:6,7,12
  \item 习题9.3:6
\end{itemize}

{\bf 【上交题】}

\begin{itemize}
  \setlength{\itemindent}{1cm}
  \item 习题9.2:13
  \item 习题9.3:8(1,2)
\end{itemize}


{\bf 【思考题】}

\begin{itemize}
  \setlength{\itemindent}{1cm}
  \item 习题9.1:12
  \item 习题9.2:10,11
  \item 习题9.3:8(3)
\end{itemize}