\begin{center}
	{\Large\bf 单元自测:数列极限}
	
	(时间:120分钟)
\end{center}

{\bf 一、计算以下极限(每题6分):}
\begin{enumerate}[(1)]
  \setlength{\itemindent}{1cm}
  \item $\limn\sqrt[n]{a^n+b^n},\;(a,b>0)$
%   \item $\limn\df{1+2!+\ldots+n!}{n!}$
  \item $\limn n\left(\df1{n^2+\pi}+\df1{n^2+2\pi}+\ldots+\df1{n^2+n\pi}\right)$
  \item $\limn\left(1-\df 1{1+2}\right)\left(1-\df
  1{1+2+3}\right)\ldots\left(1-\df 1{1+2+\ldots+n}\right)$
  \item $\limn\left[\df1{1\cdot 2\cdot 3}+\df1{2\cdot 3\cdot 4}+\ldots+
  \df1{n\cdot (n+1)\cdot (n+2)}\right]$
  \item $\limn\df{n^5}{2^n}$
%   \item $\limn\df{n}{\sqrt[n]{n!}}$
%   \item $\limn\left(1+\df 2n\right)^n$
%   \item $\limn\left(1-\df 1n\right)^n$
  \item $\limn\left(1+\df1n+\df 1{n^2}\right)^n$
\end{enumerate}

{\bf 二、证明以下极限(每题6分):}
\begin{enumerate}[(1)]
  \setlength{\itemindent}{1cm}
  \item $\limn\sqrt[n]n=1$
  \item $(n+1)^k-n^k\to 0,\;(n\to\infty)$,其中$0<k<1$
\end{enumerate}

{\bf 三、用定义证明下列极限(每题6分):}
\begin{enumerate}[(1)]
  \setlength{\itemindent}{1cm}
  \item $\limn\df{2^n}{n!}=0$
  \item $\limn\df{\sqrt{n+1}-1}{n}=0$
\end{enumerate}

{\bf 四、证明(8分):}
若$\lim\limits_{k\to\infty}a_{2k}=\lim\limits_{k\to\infty}a_{2k-1}$,
则$\{a_n\}$收敛。

{\bf 五、证明(8分):}若$\limn a_n=a\ne0$,则
$$\limn\df1{a_n}=\df1a.$$

% {\bf 五、证明(10分):}若$a_n>0\,(n\in\mathbb{N})$,$\limn a_n=a$,则$\limn\sqrt{a_n}=\sqrt
% a$。
% 
% {\bf 六、证明(10分):}
% \begin{enumerate}[(1)]
%   \setlength{\itemindent}{1cm}
%   \item $\left\{\left(1+\df 1n\right)^{n+1}\right\}$单调递减趋于$e$
%   \item $e-\left(1+\df 1n\right)^n<\df 3n$
% \end{enumerate}

{\bf 六、证明(8分):}$\left\{\sqrt[n]{n!}\right\}$发散。

% {\bf 七、证明(10分):}$\{\sin n\}$发散。

{\bf 七、(8分):}设$x_1=1$,
$$x_n=1+\df{x_{n-1}}{1+x_{n-1}},\;n=2,3,\ldots,$$
证明:$\{x_n\}$收敛,并求其极限。

{\bf 八、证明(8分):}若$a_1>0$,$a_{n+1}=a_n+\df 1{a_n},\,(n=1,2,\ldots)$,则
$$\limn\df{a_n^2}{2n}=1.$$

\newpage

\begin{center}
	{\Large\bf 解答与评分标准}
\end{center}

{\bf 一、解:}
\begin{enumerate}[(1)]
  \setlength{\itemindent}{1cm}
  \item 不妨设$a\leq b$,则
  $$b<\sqrt[n]{a^n+b^n}\leq\sqrt[n]2b,\eqno{{(+4)}}$$
  由于$\sqrt[n]2\to 1(n\to\infty)$,故由夹逼定理,原式$=b=\max\{a,b\}$.\hfill{{(+2)}}
  \item 注意到
  $$\df{n^2}{n^2+n\pi}<n\left(\df1{n^2+\pi}+\df1{n^2+2\pi}+\ldots+\df1{n^2+n\pi}\right)
  <\df{n^2}{n^2+\pi},$$
  \hfill{{(+4)}}
  
  且$\limn\df{n^2}{n^2+n\pi}=\limn\df{n^2}{n^2+\pi}=1$,故由夹逼定理,原式$=1$。\hfill{{(+2)}}
  
%   由于
%   $$\limn\df{(n+1)!}{(n+1)!-n!}=\limn\df{(n+1)!}{n\cdot
%   n!}=\limn\df{n+1}{n}=1,\eqno{{(+3)}}$$ 
%   故由Stolz定理,原式$=1$.\hfill{{(+2)}}
  \item
% %   \begin{eqnarray*}
% %   	\mbox{原式}&=&\limn\df
% %   	2{1+2}\cdot\df{2+3}{1+2+3}\ldots\df{2+\ldots+n}{1+2+\ldots+n}\\
% %   	&=&\limn\prod\limits_{k=2}^n\df{(n-1)(n+2)}{n(n+1)} \tag{12}\\ 
% %   	&=&\limn\df{(n-1)!\df{(n+2)!}{3!}}{n!\df{(n+1)!}{2!}}\\
% %   	&=&\limn\df {n+2}{3n}=\df 13
% %   \end{eqnarray*}
  \begin{align}
  	\mbox{原式}&=\limn\df
  	2{1+2}\cdot\df{2+3}{1+2+3}\ldots\df{2+\ldots+n}{1+2+\ldots+n}\notag\\
  	&=\limn\prod\limits_{k=2}^n\df{(n-1)(n+2)}{n(n+1)} \tag{{+3}}\\
  	&=\limn\df{(n-1)!\df{(n+2)!}{3!}}{n!\df{(n+1)!}{2!}}\tag{{+2}}\\
  	&=\limn\df {n+2}{3n}=\df 13\tag{{+1}}
  \end{align}
%   \item 注意到
%   $$\df{n^2}{n^2+n\pi}<n\left(\df1{n^2+\pi}+\df1{n^2+2\pi}+\ldots+\df1{n^2+n\pi}\right)
%   <\df{n^2}{n^2+\pi},$$
%   \hfill{{(+3)}}
%   
%   且$\limn\df{n^2}{n^2+n\pi}=\limn\df{n^2}{n^2+\pi}=1$,故由夹逼定理,原式$=1$。\hfill{{(+2)}}
  \item 
  \begin{align}
  	\mbox{原式}&=\df12\limn\sum\limits_{k=1}^n\left(\df 1n-\df 2{n+1}+\df
  	1{n+2}\right)\tag{{+4}}\\ 
  	&=\df12\limn\left(1-2\df 1{2}+\df 13+\df 12-2\df13+\df 14+\ldots+\df
  	1n-2\df 1{n+1}+\df 1{n+2}\right)\notag\\
  	&=\df12\limn\left(1-\df12-\df1{n+1}+\df 1{n+2}\right)=\df14\tag{{+2}}
  \end{align}
  \item
  记$a_n=\df{n^5}{2^n}$,显然$a_n>0\,(n\in\mathbb{N})$。又当$n>\df{1}{\sqrt[5]2-1}$时,
  $$\df{a_{n+1}}{a_n}=\left(\df{n+1}{n}\right)^5\cdot\df 12<1,\eqno{(+3)}$$
  故由单调有界原理,$\{a_n\}$收敛,不妨设其极限为$a$。对递推式
  $$a_{n+1}=\left(\df{n+1}{n}\right)^5\cdot\df 12a_n$$
  两端同时取极限,可得$a=\df 12a$,从而可知$\limn a_n=a=0$。\hfill(+3)
%   \item 注意到$\df{n}{\sqrt[n]{n!}}=\sqrt[n]{\df{n^n}{n!}}$,记$a_n=$
  \item 注意到
  $$\left(1+\df 1n\right)^n<\left(1+\df1n+\df 1{n^2}\right)^n<\left(1+\df
  1{n-1}\right)^n,\eqno{(+4)}$$
  而$\limn\left(1+\df 1n\right)^n=\limn\left(1+\df1{n-1}\right)^n=e$,故有夹逼定理:
  原式$=e$。\hfill(+2)
\end{enumerate}

{\bf 二、证:}
\begin{enumerate}[(1)]
  \setlength{\itemindent}{1cm}
  \item
  显然$\sqrt[n]{n}>1$。记$h_n=\sqrt[n]{n}-1\,(n\in\mathbb{N})$,以下只需证明$\limn{h_n}$=0即可。
  事实上,由
  $$n=(1+h_n)^n=1+nh_n+\df{n(n-1)}{2}h_n^2+\ldots+h_n^n>\df{n(n-1)}{2}h_n^2,\eqno{(+3)}$$ 可得
  $$0<h_n^2<\df 2{n-1}$$
  由夹逼定理可知$h_n^2\to 0(n\to\infty)$,从而$h_n\to
  0(n\to\infty)$,即证。\hfill(+3)
  \item 注意到
  $$0<(n+1)^k-n^k=n^k\left[\left(1+\df 1n\right)^k-1\right]<n^k\left[\left(1+\df
  1n\right)-1\right]=n^{k-1},\eqno{(+3)}$$
  由于$0<k<1$,故$n^{k-1}\to 0(n\to\infty)$,故由夹逼定理
  $\limn[(n+1)^k-n^k]=0$.\hfill(+3)
\end{enumerate}

{\bf 三、证:未用定义证明不得分!}
\begin{enumerate}[(1)]
  \setlength{\itemindent}{1cm}
  \item
  注意到当$n>2$时,$\df{2^n}{n!}<\df4n$,故对$\forall\e>0$,取$N=\mathrm{max}\{4/{\e},3\}$,
  {\hfill{(+4)}}\\
  则对$\forall n>N$,有
  $$\left|\df{2^n}{n!}-0\right|<\df4n<\df 4N\leq\e,\eqno{(+2)}$$
  即证。
  \item 对$\forall\e>0$,令$N=1/\e^2$,{\hfill{(+4)}}\\
  则对$\forall n>N$,有
  $$\left|\df{\sqrt{n+1}-1}{n}\right|=\df {n+1-1}{(\sqrt{n+1}+1)n}<\df1{\sqrt
  n}<\df1{\sqrt N}=\e,\eqno{(+2)}$$
  即证。
\end{enumerate}

{\bf 四、证:}
设$\lim\limits_{k\to\infty}a_{2k}=\lim\limits_{k\to\infty}a_{2k-1}=a$,
则对$\forall\e>0$,$\exists K_1>0,K_2>0$,使对$\forall k>K_1$,
$$|a_{2k}-a|<\e,$$
对$\forall k>K_2$,
$$|a_{2k-1}-a|<\e.\eqno{(+4)}$$
从而,令$N=\max\{2K_1,2K_2-1\}$,则由以上两式,对$\forall n>N$,皆有
$$|a_n-a|<\e.\eqno{(+4)}$$
即证。

{\bf 五、证:}
% 由极限的保号性可知,$a\geq 0$。\hfill(+2)
% 
% (1)若$a=0$。对$\forall\e>0$,令$\e_1=\e^2$。由$\limn a_n=0$,对$\e_1$,$\exists
% N>0$,对$\forall n>N$,有 $$|a_n-0|=a_n<\e_1,$$
% 从而
% $$|\sqrt{a_n}-0|=\sqrt{a_n}<\sqrt{\e_1}=\e,$$
% 也即$\limn\sqrt{a_n}=0=\sqrt a$。\hfill(+3)
% 
% (2)若$a>0$。由$\limn a_n=a$,对$\forall\e>0$,$\exists N>0$,对$\forall n>N$,有
% $$|a_n-a|<\e,$$
% 进而
% $$|\sqrt{a_n}-\sqrt a|=\df{|a_n-a|}{\sqrt{a_n}+\sqrt a}
% <\df {|a_n-a|}{\sqrt a}<\df 1{\sqrt a}\e,$$
% 从而$\limn\sqrt{a_n}=\sqrt a$。\hfill(+5)
由极限的保号性可知,存在$N_1$,对任意$n>N_1$,使得
$$|a_n|>\df{|a|}2.\eqno{(+3)}$$
任取$\e>0$,由$\limn a_n=a$,对$\e_1=\df{a^2}2\e$,存在$N_2$,对任意$n>N_2$,恒有
$$|a_n-a|<\e_1.\eqno{(+3)}$$
从而,对任意$n>N=\max\{N_1,N_2\}$,总有
$$\left|\df1{a_n}-\df1a\right|=\df{|a_n-a|}{|a_na|}
<\df{\e_1}{\df{a^2}2}=\e.\eqno{(+2)}$$
即证。

{\bf 六、证:}
% 参见教材第90-91页第17题。\hfill(每小题5分)
任取$a>0$。记$a_n=\df{a^n}{n!}$,显然$a_n>0$,且当$n>a-1$时,
$$\df{a_{n+1}}{a_{n}}=\df a{n+1}<1\quad\Rightarrow\quad a_{n+1}<a_n.$$
故由单调有界原理,$\{a_n\}$收敛,设其极限为$A$。\hfill(+3)

在递推式$a_{n+1}=\df a{n+1}a_{n}$
两端同时令$n\to\infty$,易得$A=0$。\hfill(+2)

取$\e=1$,由$\limn a_n=0$,可知存在$N$,使当$n>N$时,总有$a_n<\e=1$,也即
$$a^n<n!\quad\Rightarrow\quad a<\sqrt[n]{n!}.$$
由$a$的任意性,可知$\{\sqrt[n]{n!}\}$无界,从而必发散。\hfill(+3)

% {\bf 七、证法一:}用反证法。设有$\limn\sin n=a$,则
% $$\limn[\sin(n+2)-\sin n]=0.$$
% 进而由
% $$\sin(n+2)-\sin n=2\sin 1\cos(n+1),$$
% 可得$\limn\cos(n+1)=0$。又\hfill(+4)
% $$\cos(n+1)=\cos n\cos 1-\sin n\sin 1,$$
% 可得$\limn\sin n=0$。如此就有\hfill(+4)
% $$\limn\sin n=\limn\cos n=0,$$
% 显然与$\sin^2n+\cos^2n=1$矛盾。\hfill(+2)
% 
% {\bf 证法二:}注意到当$x\in[2k\pi+\pi/4,2k\pi+3\pi/4]\,(k\in\mathbb{N})$时,总有
% $$\sin x>\sqrt2/2.\eqno{(+2)}$$
% 由于此类区间的长度均超过$1$,故其中必包含至少一个自然数。于是对每个$k\in\mathbb{N}$,
% 可取自然数$n^{(1)}_k\in[2k\pi+\pi/4,2k\pi+3\pi/4]$,显然$\{\sin{n^{(1)}_k}\}$构成了$\{\sin
% n\}$的一个子列。\hfill(+2)
% 
% 若$\{\sin n\}$收敛,则$\{\sin{n^{(1)}_k}\}$也收敛,且与之极限相同。由极限的保号性,
% 可知$\{\sin{n^{(1)}_k}\}$的极限不小于$\sqrt2/2$,从而$\{\sin
% n\}$的极限应大于等于$\sqrt2/2$。\hfill(+4)
% 
% \hspace{2em}同理,利用区间$[2k\pi+5\pi/4,2k\pi+7\pi/4]\,(k\in\mathbb{N})$,可构造$\{\sin
% n\}$的另一个子列 $\{\sin{n^{(2)}_k}\}$。若$\{\sin n\}$收敛,同样利用保号性可证明其极限应小于等于$-\sqrt2/2$。
% \hfill(+2)
% 
% \hspace{2em}以上两方面的结论矛盾,故假设错误,即证。

{\bf 七、证:}对任意$n=1,2,\ldots$,显然$x_n>0$。$x_1=1<\df32=x_2$。设$x_{n-1}<x_n$,则
$$x_{n+1}-x_n=\df{x_n-x_{n-1}}{(1+x_{n-1})(1+x_n)}>0,$$
从而可知,$\{x_n\}$单调递增。\hfill(+3)

又$x_n=2-\df1{1+x_{n-1}}<2,n=1,2,\ldots$,故$\{x_n\}$有上界。由单调有界原理,$\{x_n\}$收敛,
设其极限为$A$。\hfill(+2)

在递推式两边同时令$n\to\infty$,可得
$$A=1+\df A{1+A},$$
解得$A=\df{1\pm\sqrt5}2$。由极限的保号性,易知$A\geq0$,从而$\{x_n\}$的极限为$\df{1+\sqrt5}2$。
\hfill(+3)

{\bf 八、证:}显然$\{a_n\}$为严格单调递增的正数列。假设其收敛于$a$,在递推公式两边取极限可得
$$a=a+\df1a,$$
该方程无解,故假设不成立。由此可知$\{a_n\}$必严格单调递增趋于$+\infty$,也即
$$\limn\df{1}{a_n}=0.\eqno{(+4)}$$
从而,由Stolz定理
$$\limn\df{a_n^2}{2n}=\limn\df{a_{n+1}^2-a_n^2}{2}=\df12\limn\left(2+\df1{a_n^2}\right)=1.$$
即证。\hfill(+4)
