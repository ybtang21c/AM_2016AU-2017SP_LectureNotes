{\large\bf An UNOFFICIAL Summary for Advanced Mathematics, 

Spring, 2015}
\bigskip

{\large\bf Part 1,试题分析}
\bigskip

{\bf 一、填空(15分,每题3分)}:平均10.12\ps{比秋季高$0.01$:P}

五题全错:5人!

\begin{enumerate}
  \setlength{\itemindent}{1cm}
  \item (12人错)典型错误:未写出$\d x,\d y,\d z$前的数值
  \item (25人错)极少数人把直线方程写成了平面方程
  \item (11人错)送分题,知道考什么基本就没问题
  \item (55人错,超过一半!10种错误答案!)有一定的计算量,可以考虑使用一些对称性,anyway,
  三重积分的计算始终是下册的难点之一!
  \item (66人错,对的是少数!空白的不少)典型问题:和差化积后错误展开,未展开在
  $\pi/2$处,写的是$\cos x$的展开式,只写了前$n$项(无通项公式),计算错误。
  幂级数展开还是练习得太少!
\end{enumerate}

{\bf 二、选择(15分,每题3分)}:平均11.78

\begin{enumerate}
  \setlength{\itemindent}{1cm}
  \item (13A,2C,3D)二重极限判敛的基本题型!
  \item (26A,17B,3D)Fourier级数展开的一些基本技巧(对称性、区间平移)
  \item (2A,1B,4D)送分题
  \item (9A,17B,12C)梯度的几何意义,梯度与等值线的关系,平时应该多举一些例子
  \item (3B,3C)课题上不讲,但还是要求记住有关规则!增加一些课后作业和测验题!
\end{enumerate}

{\bf 三、(6分)}:平均5.54

非常非常基本的偏导数应用!典型错误:计算错误($D$算错、分量带入错误),少写一个方程,1人空白。

{\bf 四、(6分)}:平均5.31

偏导数的计算,基本题型。典型错误:漏项(21),多项(11),算错(5),空白(2),乱写(1)。

{\bf 五、(6分)}:平均5.35

极坐标变换下的二重积分。典型错误:极坐标定限错(5),积分算错(2),空白(1),乱作(1),
直接用圆锥体积公式(1),看成了曲面积分(1)。

{\bf 六、(6分)}:平均3.53

解Bernoulli方程。典型错误:看成齐次方程(16),常数变易法(9),用全微分方程但算错(21),
代换后套一阶线性方程公式算错(6),看不出类型未动笔(25).

{\bf 七、(6分)}:平均4.10

幂级数的收敛域与求和。典型错误:步骤不清晰或无过程(2),收敛域算错(18),未求收敛域(2),
和函数算错(45),粗心(2,例如:$\ln(1+t)$与$\ln(1-t)$混淆、$C=0$算成$C=\ln2$),空白(4)

{\bf 八、(8分)}:平均2.84

全卷得分率最低的题目,基本的曲面积分,分不清类型,或者不会计算,既在意料之中(难点),
又在意料之外(错得如此之多)!{\it 曲面曲线积分的计算还要大力加强,特别是不同类型积分之间的
转换和变换}

典型错误:混淆两类积分(68,又是一大半!例如:使用Gauss公式、用第二型算不下去或算错),
积分算错或不会算(20),未动笔(22),乱作(2,例如曲面曲线积分混淆、使用Stokes公式)

{\bf 九、(8分)}:平均4.14

Green公式对于奇点的讨论,典型题目,但掌握情况很不令人满意!

典型错误:未讨论积分曲线的不同情况(83!!!),“挖洞”后曲线方向算反(2),偏导数算错(9),
第二个积分算错(8),乱作或未动(8)

{\bf 十、(8分)}:平均5.40

“补全”后使用Gauss公式计算第二型曲面积分,年年必考的题目!

典型错误:用Gauss公式后积分计算错误或不会算(57!),符号算反(31),补全后未减去多余的部分(3),
未补全(2),乱作或未动(12,例如:第二型直接化成第一型计算,使用Stokes公式),Gauss公式用错(2)

{\bf 十一、(8分)}:平均5.46

万能题,既可作为极值问题,又可作为条件极值问题,既可作为大学期末考试,又适合高考。

典型错误:计算错误(31),为求出或算错最小值点(21),未动(26),无过程(2)

{\bf 十二、(8分)}:平均4.01

莫名其妙的最后一题,有点竞赛风,和考查基础知识的掌握似乎关系不大。

典型错误:第二部分不知所云或者未动(67),全部未动(21),二维向量写成三维的(1)。

\bigskip
{\large\bf Part 2,成绩概况}

\begin{enumerate}[(1)]
  \setlength{\itemindent}{1cm}
  \item 111人全部参加考试,卷面平均分67.58,及格率74.07\%
  \item 最终成绩按照卷面80\%,平时20\%综合计算,平均分71.26,及格率96.30\%
  \item 综合成绩分布如下,90分以上3人,80-89分24人,70-79分41人,70分以下40人
  \item 综合成绩最高分92分,最低分21分
\end{enumerate}

\bigskip
{\large\bf Part 3,分析小结}

\begin{enumerate}[(1)]
  \setlength{\itemindent}{1cm}
  \item 本次考试难度维持略高(和上学期相近),平均分最高的班平均分约为84分,落下
  我们一大截
  \item 在技术类普班中,我们的成绩处于中等水平,仍为略偏下
  \item 考试答卷情况较第一学期有所改善,规范性有所提高,但看得出大家的应试心理
  还不够成熟老练(比如最后一题很多人直接选择不动笔,其实难度并不高!)
  \item 平时作业表现落后的同学稳稳垫底,说明抓好平时仍然是提高底部成绩的关键
  \item 重点的知识点在课堂上要注意特别强调,典型问题作错和不会导致丢分不少,
  教学中要认真总结
  \item 新的一届务必花大力气抓好作业、答疑和测试环节!!!
\end{enumerate}


